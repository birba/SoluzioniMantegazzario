\documentclass[a4paper,11pt]{article}

\title{Soluzioni Mantegazzario}
\author{Federico Franceschini, Dario Ascari, Dario Balboni, Umberto Pappalettera \\ 
Andrea Marino, Gianluca Tasinato}
\usepackage[utf8]{inputenc}
\usepackage{amsmath}
\usepackage{amssymb}
\usepackage{xfrac}
\usepackage[italian]{babel}
\usepackage{xifthen}
\usepackage{xparse}
\usepackage{color}
\usepackage{forloop}
\usepackage{braket}

% Definiamo i vari ambienti
\NewDocumentCommand{\ZZ}{G{}}{
  \IfNoValueTF{#1}
	{\mathbb{Z}}
	{\mathbb{Z}_{#1}}
}
\NewDocumentCommand{\FF}{G{}}{
  \IfNoValueTF{#1}
	{\mathbb{F}}
	{\mathbb{F}_{#1}}
}

\newcommand{\ex}[1]{\subsubsection*{#1}}
\newcounter{numpsi}
\newcommand{\HP}[1]{%
	\forloop{numpsi}{0}{\value{numpsi} < #1}{%
		{\color{gold}{\large$\Psi$}}%
	}%
}%

\newcommand{\su}[2]{\sfrac{#1}{#2}}

\newcommand{\PP}{\mathbb{P}}
\newcommand{\QQ}{\mathbb{Q}}
\newcommand{\NN}{\mathbb{N}}
\newcommand{\RR}{\mathbb{R}}
\newcommand{\KK}{\mathbb{K}}
%\newcommand{\FF}[1]{\mathbb{F}_{#1}}
%\newcommand{\ZZ}[1]{\mathbb{Z}_{#1}}

\newcommand{\Nplus}{\NN^{+}}

\newcommand{\cart}{\times}
\newcommand{\Mtr}[3]{\mathcal{M}(#1, #2, #3)}

\newcommand{\gen}[1]{\langle #1 \rangle}
\newcommand{\tc}{\mbox{ t.c. }}
\newcommand{\Norm}[1]{ \lVert {#1} \rVert}
\newcommand{\Zx}{\mathbb{Z}[x]}
\newcommand{\Qx}{\QQ[x]}
\newcommand{\Zp}{\su{\mathbb{Z}}{p\mathbb{Z}}}
\newcommand{\Zpx}{\Zp [x]}
\newcommand{\Hint}{{\bf Hint: }}
\newcommand{\Ker}{\mathcal{K}\mbox{er} }
\newcommand{\degree}{\mbox{deg}}

\newcommand{\MCD}[2]{\mathcal{(} #1 \mathcal{,} #2 \mathcal{)}}


\newcommand{\rec}[1]{{\bf #1}}
\newcommand{\equip}{\sim}
\newcommand{\card}{\mathbf{card}}
\newcommand{\norm}[1]{\mid{#1}\mid}
\newcommand{\todo}{{\bf TODO Prossimamente}}
\newcommand{\NINI}{{\bf NINI }}
\newcommand{\hide}[1]{{\color{white}{#1}}}
\definecolor{gold}{rgb}{0.85,0.66,0.00}

\begin{document}
\maketitle

\section*{Convenzioni}
Ciao \\
Nel seguito useremo le seguenti convenzioni: \\
\begin{itemize}
\item $\equip$ verr\`a usato per indicare l'equipotenza tra cardinalit\`a insiemistiche.
\item Verranno spesso usate $\ge$ e $\le$ per disuguaglianze tra cardinalit\`a insiemistiche.
\item La \NINI (Non-empty Intersection of Nested Intervals) dice che dato un insieme numerabile di intervalli chiusi, ciascuno contenuto nel precedente, essi hanno intersezione non vuota. Vale anche per gli intervalli aperti se l'inclusione \`e stretta. (Nota: in generale vale per i compatti)
\item Alle volte faremo uso della cosiddetta {\it ``notazione intuitiva``} ideata dal grande teorico $\mathcal{D}$.$\mathcal{C}$.$\mathcal{B}$. cioè se trovate una $n$ è un naturale (casomai nullo) ma se trovate un $1/n$ è un naturale positivo e via dicendo... 
\end{itemize}

\section{Teoria degli Insiemi}
\ex{1.30 \HP{4}} Riportiamo solo il polinomio bigettivo: $p(x,y) = \frac{1}{2}(x+y+1)(x+y)+x$. Questo polinomio conta i punti a coordinate intere sul piano per diagonali del tipo $x+y = k$.
\ex{1.31} ${\NN}^{\NN} \equip \RR$: \\ ($\ge$) Ad ogni numero reale nell'intervallo $(0,1)$ si associa la successione delle sue cifre. \\ ($\le$) Scriviamo le cifre in base due dell' $i$-esimo numero della successione nell' $i$-esima colonna di una tabella (a partire dalla cifra delle unità).  Leggiamo ora queste cifre per diagonali come le cifre dopo la virgola di un numero reale in base dieci in $[0,1)$.
\ex{1.36} Le funzioni continue da $\RR$ in $\RR$: basta conoscere i valori della funzione sui razionali ed estenderla per continuit\`a. Dunque la cardinalit\`a \`e $\RR^\QQ \equip \RR$.
\ex{1.39} Uso come lemma \rec{1.44}.
\ex{1.40} Uso come lemma la prima met\`a di \rec{1.42}.
\ex{1.41} Dimostro per prima cosa che $A \cart A \equip A$: Considero l'insieme ordinato $$\mathcal{F} := \{(f,X) \mid f: X \cart X \rightarrow X,\, X \subseteq A,\, f \mbox{ bigettiva },\, X \mbox{ infinito}\, \}$$ Osservo che \`e non vuoto. La relazione d'ordine $\preceq$ definita da $$(f,X) \preceq (g,Y) \Leftrightarrow X \subseteq Y \wedge g\mid_{X\cart X} = f$$. \\ Applico Zorn ed ottengo l'esistenza di un massimale $(h, M)$. Se $\card(A \setminus M) < \card(M)$ trovo una bigezione da $A$ in $M$. Altrimenti trovo un elemento pi\`u grande del massimale. (Non immediato)
\ex{1.42} Per la prima met\`a usate Zorn. Per il punto 2 uso il lemma \rec{1.40}.
\ex{1.43} Uso come lemma \rec{1.42} e leggo {\it per colonne e non per righe}.
\ex{1.44} Usate Zorn.
\ex{1.46} Devo dimostrare un po' di disuguaglianze tra cardinalit\`a: ogni volta sostituisco dal lato che voglio dimostrare essere maggiore $X\cart Y$ al posto di $X$ (o $Y$) (usando \rec{1.41} a palla). \\ Ad esempio: supponiamo $\card(Y) \ge \card(X)$; voglio trovare una funzione iniettiva $\Phi$ da $\{f: X \rightarrow Y\}$ in $\{g: X \rightarrow Y \mid g \mbox{ iniettiva }\}$: noto che, essendo $X \cart Y \equip Y$, vale $$\{g: X \rightarrow Y \mid g \mbox{ iniettiva }\} \equip \{g: X \rightarrow X \cart Y \mid g \mbox{ iniettiva }\}$$ Definisco $\Phi(f)$ come la funzione $g$ che manda $x \mapsto (x,f(x))$.
\ex{1.50} Supponiamo esista una funzione iniettiva da RHS in LHS: fissato un indice $i\in \mathcal{I}$, considero la controimmagine di $X_i$: l'insieme delle componenti lungo $Y_i$ di tali controimmagini non pu\`o essere tutto $Y_i$ (perch\`e $\card(X_i) < \card(Y_i)$); quindi esiste $\check{y_i} \in Y_i$ che non \`e la $i$-esima componente di nessun elemento della controimmagine di $X_i$. Dove viene mandato $\prod_{i \in \mathcal{I}} \{\check{y_i}\}$?

\section{Numeri Reali e Disuguaglianze}
\ex{2.5} Pigeonhole sull'insieme delle parti frazionarie di $a, 2a, \ldots$
\ex{2.29} 
Disuguaglianza a sinistra: usare come ipotesi di comodo $abcd=1$. Effettuare il cambio di variabili $a \rightarrow \frac {1} {a}$ e cicliche. Si conclude dopo un passaggio per per la disuguaglianza fra medie M_{1/3}\ge M_0=GM
Disuguaglianza centrale: effettuare il seguente cambio di variabile: $a\rightarrow a^6$ e cicliche.
Otteniamo $3(a^2b^2c^2+a^2b^2d^2+a^c^2d^2+b^2c^2d^2\le 2(a^3b^3+a^3c^3+a^3d^3+b^3c^3+b^3d^3+c^3d^3)$. Si noti ora che 3a^2b^2c^2\le  a^3b^3+a^3c^3+b^3c^3 per AM-GM. 
Ripetendo il procedimento per ogni addendo a LHS e sommando le disuguaglianze otteniamo proprio la tesi.
Disuguaglianza a destra: effettuare il cambio di variabile $a\rightarrow a^2$ e cicliche. Otteniamo $3(a^2+b^2+c^2+d^2)\ge ab+ac+ad+bc+bd+cd\LeftRightarrow (a-b)^2+(a-c)^2+(a-d)^2+(b-c)^2+(b-d)^2+(c-d)^2\ge 0$ che è vera.

\ex{2.30}
Legare la lunghezza delle proiezioni a quella dei rispettivi lati in una formula che indichi l'area complessiva del triangolo. 
L'espressione ottenuta è prodotto scalare fra due vettori particolari. Finire con Cauchy-Schwarz.

\section{Successioni}
\ex{3.3} Per $\norm{p} < 1$ \`e vera. Altrimenti ci sono controesempi. \hide{logaritmi o cose così}
\ex{3.4} Posto $\Delta_k:=a_k -a_{k-1}$ l' ipotesi si riscrive come: $ 2\Delta_n+\Delta_{n-1}<0$. Osservo che i due addendi non possono essere entrambi positivi. Osserviamo inoltre che preso un $\Delta_k<0$ si ha che le somme $\Delta_k+\ldots+\Delta_{k+N}<0$ perchè posso accoppiare ogni termine positivo col precedente che deve essere negativo. Questo significa che ogni volta che la successione si abbassa sotto un certo livello (fa un salto negativo) vi rimane sotto definitivamente. Cioè che $\Delta_k<0 \Rightarrow a_{k-1}>a_n$ con $n\ge k$. Posto $\ell:=\liminf_{n} a_n$ (\'e un numero reale perch\'e la successione \'e limitata inferiormente) e fissato $\epsilon>0$ definisco $I_{\epsilon}:=[\ell-\epsilon, \ell +\epsilon]$. Per definizione posso mettermi nella zona in cui la successione sta definitivamente sopra $\ell-\epsilon$ e frequentemente sotto $\ell +\epsilon$, preso qui un certo $a_k$ ci sono due casi, se $\Delta_k\ge0$ so che $\Delta_{k+1}<0$ dunque $a_{k+1}$ sta in $I_{\epsilon}$ e tutta la successione ci sta. Se invece $\Delta_{k}<0$ si ha che necessariamente $a_{k-1}\in I_{3\epsilon}$ (perchè sia $a_{k-1}$ che $a_{k-2}$ erano sopra $\ell-\epsilon$) dunque la successione sta definitivamente in $I_{3\epsilon}$. Per l' arbitrarietà di $\epsilon$ si conclude.
\ex{3.7} Stimare fattoriali (dopo essere passati al logaritmo) e serie con gli integrali o alla peggio usate brutalmente la formula di Stirling. \\ Oppure \hide{, se siete persone malvagie,} usate Stolz-Cesaro.
\ex{3.17} Mostrare che gli intervalli $I_n = [x_n, y_n]$ sono inscatolati i.e. $I_{n+1} \subseteq I_{n}$ e che l'ampiezza di tali intervalli tende a zero. (Si conclude per la \NINI).
\ex{3.18} La successione è evidentemente positiva e strettamente crescente. Supponiamo sia finito $L:=\limsup_n x_n$. Scelgo $\epsilon<1/L$ e prendo (grazie alla definizione di $\limsup$) un $L-\epsilon<x_N<L$ e scrivo:
$$ L>x_{N+1}=x_N+\frac{1}{x_N}>L-\epsilon+\frac{1}{L}$$
Guardando primo e ultimo membro si ha un assurdo. Per valutare l' ordine di crescita interpreto $x$ come funzione di $n$ variabile reale e osservo che $\frac{dx}{dn}\equip x_{n+1}-x_n$ l' equazione ricorsiva diventa: $\frac{dx}{x}=dn$ integrando ottengo soluzioni del tipo $x_n=\sqrt{n}$. Il risultato è corretto a meno di costanti moltiplicative/additive come si può verificare per induzione.

\section{Serie Numeriche}
\ex{4.5} Usare la formula per $\tan(\alpha - \beta)$ e telescopizzzare.
\ex{4.13} Moltiplicate per 3 $\ldots$ \hide{\`e il cubo di un binomio telescopico}
\ex{4.17} Usate la sommazione per parti di Abel, ovvero \rec{1.25}. (Non \`e affatto inutile come sembra)
\ex{4.19} Usate nuovamente la sommazione per parti di Abel (\rec{1.25})
\ex{4.21} Ancora sommazione per parti (\rec{1.25})
\ex{4.22} Pongo $\frac{b_n}{a_n} = 1 + \varepsilon_n$: serve che $\sum (-1)^n a_n$ converga e $\sum (-1)^n \varepsilon_n a_n$ diverga. Trovare tali $a_n$ e $\varepsilon_n$.
\ex{4.27} Osservare che $\frac{1}{n-1} = \sum_{j \ge 1} \frac{1}{n^j}$ e fattorizzare.
\ex{4.32} Dare la formula chiusa per $\sum_{k=1}^{n} \sin(k)$ scrivendola come geometrica di esponenziali complessi. Poi usare sommazione per parti di Abel (\rec{1.25}).

\section{Topologia di $\RR$}
\ex{5.12} Usando \rec{5.16} si ha che $\card(\mbox{aperti}) \le \card\left({\left(\RR \cart \RR\right)}^{\NN}\right) = \card(\RR)$.
\ex{5.15} Controesempio alla seconda domanda: $A = \{n+\frac{1}{n} \mid n \in \NN^{*}\}$, $B = \ZZ$.
\ex{5.16} $A$ aperto di $\RR$. Definiamo $\forall a \in A \quad I_a := \bigcup \{(x,y) \mid a \in (x,y) \subseteq A \}$; gli $I_a$ sono intervalli e partizionano $A$. Una famiglia di intervalli disgiunti di $\RR$ ha cardinalit\`a al pi\`u numerabile; per provarlo intersecarli con $\QQ$ oppure osservare che per ogni lunghezza positiva fissata $\ell$ ve ne sono al pi\`u una quantit\`a numerabile di lunghezza maggiore di $\ell$. A questo punto si sceglie $\ell = \frac{1}{n}$ e si numerano.
\ex{5.17} Passare al complementare e usare \rec{5.16}
\ex{5.18} Usare la \NINI per dire che esistono punti che non vengono coperti dai chiusi del ricoprimento. (passare al complementare ed usare la \NINI sugli aperti)
\ex{5.19} Passare al complementare e usare \rec{5.20}
\ex{5.20} \NINI
\ex{5.21} Contarli come in \rec{5.16}
\ex{5.26} Caratterizzare i chiusi come gli insiemi che contengono i loro punti di accumulazione. 
\ex{5.23} Bisezionare ed ogni volta e scegliere negli intervalli creati un qualsiasi elemento (se c'è) di $F$. (si generalizza ad $\RR^{n}$)

\ex{5.30} Seconda parte: supponiamo per assurdo $\card(A) > \card(\NN)$. Operiamo una bisezione su $A$ per vedere dove ci sono una quantit\`a pi\`u che numerabile di punti: ad ogni bisezione sono ad un bivio e la scelta pu\`o essere {\it libera} o {\it obbligata}. Pu\`o essere {\it libera} se da entrambi i lati ci sono una quantit\`a pi\`u che numerabile di punti. Non si possono presentare definitivamente scelte {\it obbligate}: qualunque successione di scelte {\it libere} io faccia prima o poi mi si presenter\`a un'altra scelta {\it libera} (se no in quell'intervallo avrei solo una quantit\`a numerabile di punti). Ma allora ho libert\`a $\card(\NN)$ volte di scegliere tra due possibilit\`a: almeno $\card(2^{\NN}) \equip \card(\RR)$ punti di accumulazione.\\
Alternativa: considero i punti di $A$. Aut sono isolati (dunque hanno cardinalità al più numerabile) aut sono di accumulazione (dunque stanno in $A'$ e hanno ancora cardinalità al più numerabile). 

\ex{5.31}Prima parte: consideriamo $A_0:=\{1/n:n\in\NN^{*}\}$ quest' insieme chiaramente si accumula in $0$ ed è composto da soli punti isolati. Costruiamo $A_1$ traslando e omotetizzando copie di $A_0$ in modo che si accumulino sui punti del tipo $1/n$ e che si abbia quindi $A_1'=A_0$. Ricorsivamente si riesce ad ottenere la tesi. Seconda parte: incollare nell'intervallo $[2n, 2n+1]$ la soluzione della prima parte con $n$.

\ex{5.32} Sia $X$ il mio insieme. Considero un suo punto $x_o$: trovo una successione di punti di $X$ che tende a $x_o$; reitero il procedimento per ogni punto di tale successione e cos\`i via: ho trovato $\card(\NN^{\NN})$ punti.

\section{Spazi metrici, normati e topologici}
\ex{6.1} Osservo che dato un $c\in C_1$ deve esistere $r>0$ tale che $\overline{B_{r}}(c)\cap C_2=\emptyset$. Se così non fosse $c$ sarebbe di accumulazione per $C_2$, dunque vi apparterrebbe, assurdo per disgiunzione. Unendo tali $B_r$ ottengo il mio aperto.
\ex{6.4} Scrivo il mio aperto come unione numerabile di palle aperte: $A=\bigcup_{n\in\NN}B_{r_n}(x_n)$. Ogni palla aperta si scrive come successione strettamente crescente di palle compatte: $B_{r_n}(x_n)=\bigcup_{k\in\NN}Y_{n,k}$ dove abbiamo posto $Y_{n,k}:=B_{ \frac {k}{k+1}\, r_n}(x_n)$. Pongo $\displaystyle K_n:=\bigcup_{k=1}^n Y_{k,n-k}$ e funziona.
\ex{6.5} Controesempio: mettere quattro punti su una sfera con la metrica delle geodetiche.
\ex{6.7} Controesempio $X=\RR$, $A=\{n+1/n\}$, $B=\NN$.
\ex{6.13} Se una sottosuccessione \( (x_{k(n)})\) converge a \(x_{\infty}\), allora per ogni \(n\) ed $m$ vale: 
\[ d(x_{\infty},x_n) \le d\left({x_{\infty}, x_{k(m)}}\right) + d\left({x_{k(m)},x_n}\right) \]
che è infinitesimo per la convergenza di \( \left({x_{k(n)}}\right)\) e il fatto che la successione sia di Cauchy ( \(k(m)\) è un naturale \(\ge n\) ).
\ex{6.27} Supponiamo che, nelle ipotesi date, $K$ non sia connesso. Allora $K$ si spezza nell' unione di due aperti disgiunti $A_1$ e $A_2$. Osservo che definitivamente si deve avere $K_n \cap A_1 \neq \emptyset$ e $K_n \cap A_2 \neq \emptyset$ altrimenti frequentemente $\delta (K_n, K)>r$ dove $r$ è il raggio (fissato) di una palla centrata in un certo $k\in K$ tale che $B_r(k)\subseteq A_1$ (usiamo il fatto che se esiste una palla di raggio $r$ centrata in $k\in K$ che è sempre disgiunta da $K_n$ allora $\delta(K,K_n)\geq r$). A questo punto esiste una successione di $x_n$ tali che $x_n\in K_n\setminus (A_1 \cup A_2)\subseteq (A_1 \cup A_2)^c$. Questa successione per il \rec{6.22} converge ad un elemento $x_{\infty}\in K$; ma $(A_1 \cup A_2)^c$ è chiuso e quindi contiene i limiti delle proprie successioni, assurdo.
\ex{6.37} "Solo se". Supponiamo vera l' identità del parallelogramma. L' unica definizione sensata a posteriori è:
$$
2\langle x | y\rangle :=\Norm{x+y}^2-\Norm{x}^2-\Norm{y}^2=\Norm{x}^2+\Norm{y}^2-\Norm{x-y}^2=\frac{1}{2}\Norm{x+y}^2-\frac{1}{2}\Norm{x-y}^2
$$
Le proprietà immediate da verificare (usando l' equivalenza delle definizioni) sono:
$$
\langle x |x \rangle=\Norm{x}^2,\qquad  \langle x | -y \rangle = - \langle x | y \rangle, \qquad \langle x+y|y\rangle=\langle x|y\rangle +\Norm{y}^2,\qquad \langle x |y \rangle=\langle y |x \rangle
$$
Usando la subadditività della norma poi:
$$
\norm{\langle x|y\rangle} \leq \Norm{x}\Norm{y}
$$
Da questa disuguaglianza si ha che la funzione definita è continua su $V\cart V$.
Proviamo l' additività. Scriviamo la seguente espressione:
$$
\Norm{x+2z}^2=\Norm{(x+z)+z}^2=\Norm{x}^2+2\Norm{z}^2+2\langle x| z\rangle +2\langle x+z|z\rangle=\Norm{x}^2+4\Norm{z}^2+4\langle x| z\rangle\quad (\star)
$$
Scrivo ora la tesi:
$$
4\langle x+y |z\rangle=4\langle x|z \rangle + 4\langle y|z \rangle
$$
Usando la terza uguaglianza della definizione e portando le cose dello stesso segno dalla stessa parte ottengo:
$$
\Norm{x+y+z}^2+\Norm{x-z}^2+\Norm{y-z}^2=\Norm{x+y-z}^2+\Norm{x+z}^2+\Norm{y+z}^2
$$
Ora moltiplico per due e applico l' identità del paralleogramma ai primi due addendi di ogni membro, ottenendo:
$$
\Norm{2x+y}^2+\Norm{y+2z}^2+2\Norm{y-z}^2=\Norm{2x+y}^2+\Norm{y-2z}^2+2\Norm{y+z}^2
$$
$$
\Leftrightarrow \Norm{y+2z}^2+2\Norm{y-z}^2=\Norm{y-2z}^2+2\Norm{y+z}^2
$$
da qui si conclude usando $(\star)$ per aprire le norme.
Manca adesso l' omogeneità. Si ha subito per omogeneità della norma che:
$$
\langle \lambda x|\lambda y\rangle= \lambda ^2 \langle x|y\rangle  \qquad \lambda\in \RR
$$
usando induttivamente l' additività ottengo che:
$$
\Braket{\frac{a}{b}\, x|\frac{c}{d}\, y}=\frac{1}{b^2d^2}\left\langle ad\, x|cb\, y\right\rangle=\frac{ac}{bd}\left\langle x|y\right\rangle\qquad a,b,c,d \in \NN
$$
questo prova l' omogeneità sui razionali, per continuità ciò è vero su $\RR$. 
\ex{6.39} Essendo $V$ a dimensione finita esiste, fissata una base, un isomorfismo $\psi$ con un certo $\RR^d$:
$$
\psi: \ V\ni\lambda_1\vec{e_1}+\ldots+\lambda_d\vec{e_d}\ \mapsto\  (\lambda_1,\ldots,\lambda_d)\in\RR^d
$$
Pongo ora su $V$ la norma  {\it euclidea}:
$$
\Norm{v}_{\mathcal{E}}:=\max_i\{\norm{\lambda_i}\}
$$
mostriamo che $B_{\mathcal{E}}$, la palla unitaria, è compatta per successioni. Data una successione di vettori unitari in $V$ le loro componenti sono $d$ successioni di reali in $[-1,1]$ dunque per Bolzano-Weierstrass hanno una sottosuccessione convergente. Considero ora un' altra generica norma su $V$ e la penso come funzione dalla palla unitaria in $\RR$:
$$\Norm{\cdot}\, : B_{\mathcal{E}}\rightarrow \RR$$
Questa funzione è continua rispetto alla metrica indotta da $\Norm{\cdot}_{\mathcal{E}}$ infatti si conta:
$$
\Norm{u-v}\leq\sum_{i=1}^{d}\norm{u_i-v_i}\Norm{\vec{e_i}}\leq d\cdot\max_i\{\norm{u_i-v_i}\}\cdot \max_i\{\Norm{\vec{e_i}}\}=K \Norm{u-v}_{\mathcal{E}}
$$
Quindi per Weierstrass ammette $\max=M$ e $\min=m$. Dunque si ha, per un generico $u\in B_{\mathcal{E}}$: 
$$m\Norm{u}_{\mathcal{E}}\leq \Norm{u}_2\leq M\Norm{u}_{\mathcal{E}}$$
Abbiamo mostrato che ogni norma è equivalente a $\Norm{\cdot}_{\mathcal{E}}$; e poichè relazione di bi-lipschitz equivalenza gode della proprietà transitiva abbiamo finito.\\ 
Per la seconda domanda si prenda $V$ come l' insieme delle successioni a valori reali assolutamente sommabili, le seguenti norme non sono equivalenti:
$$
\Norm{x_n}_1=\sup_k \norm{x_k} \quad \mbox{      e       }\quad \Norm{x_n}_2=\sum_{k\in\NN} 2^{-k}\norm{x_k}
$$
in effetti non inducono neppure la stessa topologia poichè esistono successioni di vettori di $V$ che rispetto alla prima non convergono mentre rispetto alla seconda si. (ad esempio $(x_n)_k=\delta_{n,k}$).
\ex{6.40} {\it Controesempio:} nello spazio $V$ delle successioni la cui somma dei valori assoluti converge considero il sottospazio $W$ delle successioni definitivamente nulle. Se prendo le successioni del tipo $x_n=2^{-n}\chi_{[0,k]}$ queste vivono in $W$ ma il loro limite (rispetto a $k$ si intende, cioè la successione $y_n=2^{-n}$) chiaramente no. 
\ex{6.41} {\it Completezza $\Rightarrow$ Convergenza assoluta.} Poichè la serie converge il termine generale è infinitesimo, per completezza di $\RR$ questo vuol dire che, posto $S_k=\sum_{i=0}^k x_i$, si ha:
$$
\epsilon>\sum_{i=M}^{N}\Norm{x_i}\geq \Norm{\sum_{i=M}^{N}x_i}=\Norm{S_N-S_M}\qquad\quad N,M \geq n_{\epsilon}
$$
Dunque la successione $S_N\in V$ e di Cauchy e per completezza converge.
{\it Convergenza assoluta $\Rightarrow$ Completezza.} Supponiamo $V$ incompleto. Esiste dunque $(x_n)$ che ha la proprietà di Cauchy ma non converge in $V$. Usando nella definizione di Cauchy $\epsilon=2^{-j}$ estraggo una sottosuccessione $(x_{n(j)})$ tale che la successione di partenza stia definitivamente in $B_{2^{-j}}(x_{n(j)})$ e scelgo sempre $x_{n(j+1)}\in B_{2^{-j}}(x_{n(j)})$. Osservo che anche $(x_{n(j)})$ ha la proprietà di Cauchy e per $\rec{6.13}$ neanche lei converge. Posto $y_j:=x_{n(j+1)}-x_{n(j)}$ ho che:
$$
\sum_{j=1}^{\infty}\Norm{y_j}\leq \sum_{j\in\NN}2^{-j}=1\quad \Rightarrow\quad x_{n(k)}-x_{n(0)}=\sum_{i=1}^{k}y_j\rightarrow \ell\ \  \Rightarrow\ \  x_{n(k)}\mbox{ converge}
$$ 
ma questo è assurdo.
\ex{6.50} Sia dato un ricoprimento aperto $\{A_{\lambda}\}_{\lambda\in\Lambda}$. Supponiamo che l' $\inf$ dei raggi sia $0$. Questo significa che per ogni $\epsilon>0$ esiste una palla $B_{\epsilon}(x_{\epsilon})$ che non è contenuta interamente in nessun aperto del ricoprimento. Per ogni $n$, mi gioco questa proprietà con $\epsilon=1/n$ e definisco come $x_n$ il centro della palla $B_{\epsilon}(x_\epsilon)$. Questa successione ammette una sottosuccessione convergente $(x_{n_k})$ ad un certo elemento $x_{\infty}\in X$, ma questo punto limite deve stare in un certo $A_{\lambda}$ (perchè sono un ricoprimento); dunque esiste $r>0$ tale che $B_r(x_{\infty})\subseteq A_{\lambda}$. In particolare ci saranno infiniti punti della mia sottosuccessione nell' intorno $B_{r/2}(x_{\infty})$. Ma allora scelgo un elemento della sottosuccessione abbastanza avanti in modo che $n_k>2/r$, ottengo quindi che $$B_{1/{n_k}}(x_{n_k})\subseteq B_r(x_{\infty})\subseteq A_{\lambda}$$
ma questo è assurdo perchè viola la proprietà caratteristica di $x_{n_k}$ (ha la palla di raggio dato interamente contenuta in un aperto del ricoprimento).
\ex{6.51} (\NINI in spazi metrici) Scelgo una successione tale che $x_n \in F_n$. Tale successione vive in $F_1$ che è compatto dunque qui ammette una sottosuccessione convergente a $x_{\infty}$. Per ogni naturale $N$ osservo che la sottosuccessione vive definitivamente in $F_N$ e per compattezza qui vive il suo limite $x_{\infty}$, ciò prova che $x_{\infty}\in \bigcap_{n\in \NN}F_n$.
\ex{6.54} {\it Compattezza per ricoprimenti} $\Rightarrow$ {\it Compattezza sequenziale}. \\
Supponiamo per assurdo che esista una successione $(x_n)_{n\in\NN}$ che non ammette sottosuccessioni convergenti. Questo significa che l' immagine di $(x_n)_{n\in\NN}$ non ha punti di accumulazione, cioè che:
$$ \forall x\in X\  \exists \epsilon_x > 0: \; \left( B_{\epsilon_x}(x)\setminus \{x\}\right) \cap (x_n)_{n\in\NN} = \emptyset $$
Chiaramente $\bigcup_{x\in X}B_{\epsilon_x}(x)\supseteq X$ dunque $\{B_{\epsilon_x}(x)\}_{x\in X}$ è un ricoprimento aperto; ma per ipotesi deve ammettere un sottoricoprimento finito che avrà centri $y_1,\ldots,y_k$. Per pigeonhole per almeno un $1\leq i\leq k$ ho che $B_{\epsilon_{y_i}}(y_i)$ contiene infiniti termini della successione, ma per costruzione ne può contenere al più uno. Assurdo.\\
{\it Compattezza sequenziale} $\Rightarrow$ {\it Totale limitatezza}.\\
Fissato un $\epsilon > 0$ e scelto un $x_0 \in X$ definisco ricorsivamente una sequenza $(x_n)$ scegliendo ogni termine in modo che:
$$x_{n} \in X \setminus \bigcup_{i=0}^{n-1}B_\epsilon(x_i)$$
Opero una dicotomia ({\it cit}):
\begin{itemize}
\item Se esiste $N$ tale che $X \setminus \bigcup_{i=0}^{N}B_\epsilon(x_i) = \emptyset $ ho trovato un ricoprimento finito;
\item Se un tale $N$ non esiste allora ho ottenuto una successione $(x_n)$. Per compattezza deve avere una sottosuccessione convergente (dunque di Cauchy): ma questo è assurdo perché comunque presi $m>n$ ho che:
$$
x_m\notin B_{\epsilon}(x_n)\ \Rightarrow\ d(x_m,x_n)>\epsilon
$$
\end{itemize}
{\it Compattezza sequenziale} $\Rightarrow$ {\it Completezza}.\\
Sia $(x_n)$ una successione di Cauchy; per compattezza ammette una sottosuccessione convergente ma, essendo di Cauchy $(x_n)$ converge per $\rec{6.13}$.\\
{\it Completezza} $\wedge$ {\it Totale limitatezza} $\Rightarrow$ {\it Compattezza sequenziale}.\\
Prendo una qualsiasi successione $(x_n)\subseteq X$; voglio definire ricorsivamente una sua sottosuccessione convergente; per completezza mi basta farla di Cauchy. Per totale limitatezza posso ricoprire tutto $X$ con numero finito di palle di raggio arbitrariamente fissato (diciamo $r_0=1$). La mia successione dovrà cadere infinite volte in almeno una di queste palle (per pigeonhole); ne prendo una e la chiamo $B_0$ e, poichè un sottoinsieme di un totalmente limitato è totalmente limitato, anche $B_0$ è totalmente limitato. Ma $B_0$ contiene infiniti termini della successione quindi, dato un qualunque suo ricoprimento finito posso trovare una $B_1\subset B_0$ di raggio arbitrario ($r_1=1/2$) che contiene ancora infiniti elementi della successione. Definisco così induttivemente i $B_k$ e li scelgo ogni volta di raggio $2^{-k}$. Definisco ora la mia sottosuccessione $(x_{n(k)})_{k\in\NN}$ così:
$$
n(k):=\min\left\{j\in\NN\mid j>n(k-1)\, \wedge\, x_j\in B_k\right\}
$$ 
perchè l' insieme di cui prendo il minimo non è mai vuoto. Ma la sottosuccessione così definita è chiaramente di Cauchy.\\
{\it Totale limitatezza $\wedge$ Compattezza sequenziale $\Rightarrow$ Compattezza per ricoprimenti}\\
Sia dato un ricoprimento aperto $\{A_{\lambda}\}_{\lambda\in\Lambda}$. Per compattezza sequenziale esiste un {\it numero di Lebesgue $\rho$} positivo (\rec{6.50}). Per totale limitatezza esiste un ricoprimento finito con palle di raggio $\rho$; sia $\{B(x_1),\ldots,B(x_n)\}$ tale ricoprimento finito. Per definizione di numero di Lebesgue ho che ognuna di queste palle $B(x_i)$ è contenuta in almeno un aperto $A_{\lambda_i}$ del mio ricoprimento; si ha dunque che $\{A_{\lambda_1},\ldots,A_{\lambda_n}\}$ è ancora un ricoprimento del mio insieme ed è chiaramente finito.\\
{\it Totale limitatezza} $\Rightarrow$ {\it Separabilità}.\\
Voglio costruire un sottoinsieme denso e numerabile. Per ogni $\epsilon_k=2^{-k}$ ho un ricoprimento finito di $\epsilon_k-$palle. Sia $C_k=\{c_1,\ldots,c_{n(k)}\}$ l' insieme dei centri di queste palle. L' insieme $C=\bigcup_{k\in\NN}C_k$ è numerabile (in quanto unione numerabile di insiemi finiti) e denso, infatti dato un qualsiasi $x\in X$ ed $\epsilon>0$ so che $x$ è contenuto in una palla del mio $\epsilon-$ricoprimento finito, cioè dista meno di $\epsilon$ dal un centro di queste palle, che è un elemento di $C$. Per l'arbitrarietà di $\epsilon$ ogni palla centrata in $x$ interseca $C$.

\section{Continuità}
\ex{7.40}Controsempio cannonoso: Fissiamo una base di Hamel di $\RR$ come spazio vettoriale su $\QQ$. Ora ogni reale $x$ si scrive come combinazione lineare {\it finita}: $x=\sum_{i=0}^{n_x} \frac{p_i}{q_i}\, b_i$. Definisco:
$$
f:x\mapsto \frac{1}{\gcd(p_1,\ldots,p_{n_x})}
$$ 
questa $f$ è effettivamente un controesempio.\\
Controesempio alternativo: sia $A=\{\pi,\pi^2,\ldots\}$ definisco $f$ costantemente $1$ su $A$ e nulla altrove. Fissato $\check{x}$ la sua progressione aritmetica casca al più una volta in $A$ (che si riesca anche a fare continua sulla base di questa?)...\\
Se $f$ è uniformemente continua invece il limite di $f$ è nullo. Prendo $L=\limsup f$ e so che frequentemente c'è qualcuno (sia $x_k$) sopra $L-\epsilon$. Prendo il $\delta$ dell' uniforme continuità e ho che $f([x_k-\delta,x_k+\delta])\ge L-2\epsilon$. Scelgo ora $\check{x}<2\delta$ e applico la prima ipotesi ottenendo un assurdo perchè capito negli intorni di tutti gli $x_k$ ed ho $\limsup_n n\check{x}\ge L-2\epsilon>0$ per l' arbitrarietà di $\epsilon$.
 
\end{document}
