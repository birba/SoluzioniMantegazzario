\section{Numeri Reali e Disuguaglianze}

\ex{2.5} Pigeonhole sull'insieme delle parti frazionarie di $a, 2a, \ldots$

\ex{2.13}
AM-GM.
Induzione. Il passo base per $n=2$ si verifica facilmente. Assumiamo ora che la disuguaglianza sia vera per $n$ termini e mostriamola per $n+1$.
$$\left(\prod _{i=0} ^{n+1} a_i\right)^{1/n+1}=\left(\prod _{i=0} ^{n} a_i\right)^{1/n+1}\cdot a_{n+1}^{1/n+1}=\left(\prod _{i=0} ^{n} a_i^{1/n}\right)^{n/n+1}\cdot a_{n+1}^{1/n+1}$$.
Si noti ora che gli esponenti dei due fattori sommano ad 1. Applichiamo dunque Young.
$$\left(\prod _{i=0} ^{n} a_i^{1/n}\right)^{n/n+1}\cdot a_{n+1}^{1/n+1}\le \frac {n} {n+1} \prod _{i=0} ^{n} a_i^{1/n}+\frac {1} {n+1} a_{n+1}\le \left(\sum _{i=0} ^{n}a_i + a_{n+1}\right)\cdot \frac{1}{n+1}=AM$$ dove nell'ultima disuguaglianza si è fatto uso dell'ipotesi induttiva di AM-GM su $n$ termini\\
HM-GM.
Anche qui si procede come prima cercando un modo furbo di applicare Young\\
QM-AM
Il passo base per $n=2$ è verificato. La tesi equivale a mostrare $$\sum_{i=0}^{n+1}a_i^2 \ge \frac {1}{n+1}\left(\left(\sum_{i=0}^{n}a_i\right)+a_{n+1}\right)^2= \frac {1}{n+1}\left(\left(\sum_{i=0}^{n}a_i\right)^2+a_{n+1}^2+2a_{n+1}\left(\sum_{i=0}^{n}a_i\right)\right)$$
Moltiplicando ambo i membri della disuguaglianza e spezzando alcuni addendi abbiamo $$n\sum_{i=0}^{n+1}a_i^2+\sum_{i=0}^{n}a_i^2+na_{n+1}^2\ge \left(\sum_{i=0}^{n}a_i^2\right)^2+2a_{n}\sum_{i=0}^{n}a_i^2$$
Si noti ora che $$\sum_{i=0}^{n}a_i^2 +na_{n+1}^2\ge 2a_{n+1}\sum_{i=0}^{n}a_i$$ in quanto portando a sinistra la sommatoria otteniamo la somma di $n$ quadrati
Ma $$n\sum_{i=0}^{n}a_i^2\ge(\sum_{i=0}^{n}a_i)^2$$ per ipotesi induttiva. Dunque tutti gli addendi a LHS maggiorano quelli a RHS e dunque la tesi è verificata.

\ex{2.15}
La disuguaglianza a sinistra è ovvia. Per mostrare la disuguaglianza a destra si riscrive come $\frac{x+n-1}{n}>x^{1/n}$ che è AM-GM sulla n-upla costituita da $x$ e $n-1$ volte 1.

\ex{2.22}
Riscriviamo come $$\left(\sum_{i=0}^na_i\right)^2\le\sum_{i=0}^{n}a_i^2\Leftrightarrow 2\sum_{i\neq j}^{n}a_ia_j\le (n-1)\left(\sum_{i=0}^{n}a_i^2\right)$$ 
che può essere riscritto portando RHS a sinistra come $$\sum_{i=0,\, j\neq i}^{n} (a_i-a_j)^2$$ ossia come somma dei quadrati delle $n(n-1)$ differenze. Pertanto la tesi è verificata.

\ex{2.24}
La tesi equivale a $(\frac{1}{n}\sum_{i=0}^na_i)^p\le \frac{1}{n}\sum_{i=0}^na_i^p$.
Dato che la disugaglianza è omogenea prendiamo come ipotesi di comodo $$\left(\frac{1}{n}\sum_{i=0}^na_i\right)^p=1\Rightarrow \frac{1}{n}\sum_{i=0}^na_i=1$$Effettuiamo il seguente cambio di variabile: $a_i\rightarrow 1+b_i$, con $\sum_{i=0}^nb_i=0,b_i\ge 0$. Otteniamo $$\frac{1}{n}\sum_{i=0}^n\left(1+b_i\right)^p\ge 1$$ che è vera applicando ad ogni addendo di LHS la disuguaglianza di Bernoulli con le ipotesi sulla somma descritte in precedenza.

\ex{2.29} 
{\it Disuguaglianza a sinistra:} usare come ipotesi di comodo $abcd=1$. Effettuare il cambio di variabili $a \rightarrow \frac {1} {a}$ e cicliche. Si conclude dopo un passaggio per per la disuguaglianza fra medie $M_{1/3}\ge M_0=GM$.\\
Disuguaglianza centrale: effettuare il seguente cambio di variabile: $a\rightarrow a^6$ e cicliche.
Otteniamo $$3(a^2 b^2 c^2+a^2 b^2 d^2+a^2 c^2 d^2+b^2 c^2 d^2)\le 2(a^3 b^3+a^3 c^3+a^3 d^3+b^3 c^3+b^3 d^3+c^3 d^3)$$Si noti ora che $$3a^2 b^2 c^2\le  a^3b^3+a^3c^3+b^3c^3$$ per AM-GM. 
Ripetendo il procedimento per ogni addendo a LHS e sommando le disuguaglianze otteniamo proprio la tesi.\\
{\it Disuguaglianza a destra:} effettuare il cambio di variabile $a\rightarrow a^2$ e cicliche. Otteniamo $$3(a^2+b^2+c^2+d^2)\ge ab+ac+ad+bc+bd+cd$$ $$\Leftrightarrow (a-b)^2+(a-c)^2+(a-d)^2+(b-c)^2+(b-d)^2+(c-d)^2\ge 0$$ che è vera.

\ex{2.30}
Legare la lunghezza delle proiezioni a quella dei rispettivi lati in una formula che indichi l'area complessiva del triangolo. 
L'espressione ottenuta è prodotto scalare fra due vettori particolari. Finire con Cauchy-Schwarz.

