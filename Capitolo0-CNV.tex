\textbf{\Large Convenzioni}
\vspace{1em}

Nel seguito useremo le seguenti convenzioni:

\begin{itemize}
\item $\equip$ verrà usato per indicare l’equipotenza tra cardinalità insiemistiche.
\item Verranno spesso usate $\ge$ e $\le$ per disuguaglianze tra cardinalità insiemistiche.
\item La \NINI (Non-empty Intersection of Nested Intervals) dice che, data
una successione di intervalli chiusi e limitati, ciascuno contenuto nel
precedente, essi hanno intersezione non vuota. (Nota: in generale vale per i
compatti)
\item Alle volte faremo uso della cosiddetta \emph{“notazione intuitiva“} ideata dal
grande teorico $\mathcal{D.C.B.}$, cioè: se trovate una n è un naturale (eventualmente
nullo), ma se trovate un $1/n$, $n$ è un naturale positivo, e via dicendo...
\end{itemize}