\section{Spazi metrici, normati e topologici}

\ex{6.1} Osservo che dato un $c\in C_1$ deve esistere $r>0$ tale che $\overline{B_{r}}(c)\cap C_2=\emptyset$. Se così non fosse $c$ sarebbe di accumulazione per $C_2$, dunque vi apparterrebbe, assurdo per disgiunzione. Unendo tali $B_r$ ottengo il mio aperto.

\ex{6.4} Scrivo il mio aperto come unione numerabile di palle aperte: $A=\bigcup_{n\in\NN}B_{r_n}(x_n)$. Ogni palla aperta si scrive come successione strettamente crescente di palle compatte: $B_{r_n}(x_n)=\bigcup_{k\in\NN}Y_{n,k}$ dove abbiamo posto $Y_{n,k}:=B_{ \frac {k}{k+1}\, r_n}(x_n)$. Pongo $\displaystyle K_n:=\bigcup_{k=1}^n Y_{k,n-k}$ e funziona.

\ex{6.5} Controesempio: mettere quattro punti su una sfera con la metrica delle geodetiche.

\ex{6.7} Controesempio $X=\RR$, $A=\{n+1/n\}$, $B=\NN$.

\ex{6.13} Se una sottosuccessione \( (x_{k(n)})\) converge a \(x_{\infty}\), allora per ogni \(n\) ed $m$ vale: 
\[ d(x_{\infty},x_n) \le d\left({x_{\infty}, x_{k(m)}}\right) + d\left({x_{k(m)},x_n}\right) \]
che è infinitesimo per la convergenza di \( \left({x_{k(n)}}\right)\) e il fatto che la successione sia di Cauchy ( \(k(m)\) è un naturale \(\ge n\) ).

\ex{6.27} Supponiamo che, nelle ipotesi date, $K$ non sia connesso. Allora $K$ si spezza nell' unione di due aperti disgiunti $A_1$ e $A_2$. Osservo che definitivamente si deve avere $K_n \cap A_1 \neq \emptyset$ e $K_n \cap A_2 \neq \emptyset$ altrimenti frequentemente $\delta (K_n, K)>r$ dove $r$ è il raggio (fissato) di una palla centrata in un certo $k\in K$ tale che $B_r(k)\subseteq A_1$ (usiamo il fatto che se esiste una palla di raggio $r$ centrata in $k\in K$ che è sempre disgiunta da $K_n$ allora $\delta(K,K_n)\geq r$). A questo punto esiste una successione di $x_n$ tali che $x_n\in K_n\setminus (A_1 \cup A_2)\subseteq (A_1 \cup A_2)^c$. Questa successione per il \rec{6.22} converge ad un elemento $x_{\infty}\in K$; ma $(A_1 \cup A_2)^c$ è chiuso e quindi contiene i limiti delle proprie successioni, assurdo.

\ex{6.37} "Solo se". Supponiamo vera l' identità del parallelogramma. L' unica definizione sensata a posteriori è:
$$ 2\langle x | y\rangle :=\Norm{x+y}^2-\Norm{x}^2-\Norm{y}^2=\Norm{x}^2+\Norm{y}^2-\Norm{x-y}^2=\frac{1}{2}\Norm{x+y}^2-\frac{1}{2}\Norm{x-y}^2 $$
Le proprietà immediate da verificare (usando l' equivalenza delle definizioni) sono:
$$ \langle x |x \rangle=\Norm{x}^2,\qquad  \langle x | -y \rangle = - \langle x | y \rangle, \qquad \langle x+y|y\rangle=\langle x|y\rangle +\Norm{y}^2,\qquad \langle x |y \rangle=\langle y |x \rangle $$
Usando la subadditività della norma poi:
$$ \norm{\langle x|y\rangle} \leq \Norm{x}\Norm{y} $$
Da questa disuguaglianza si ha che la funzione definita è continua su $V\cart V$.
Proviamo l' additività. Scriviamo la seguente espressione:
$$ \Norm{x+2z}^2=\Norm{(x+z)+z}^2=\Norm{x}^2+2\Norm{z}^2+2\langle x| z\rangle +2\langle x+z|z\rangle=\Norm{x}^2+4\Norm{z}^2+4\langle x| z\rangle\quad (\star) $$
Scrivo ora la tesi:
$$ 4\langle x+y |z\rangle=4\langle x|z \rangle + 4\langle y|z \rangle $$
Usando la terza uguaglianza della definizione e portando le cose dello stesso segno dalla stessa parte ottengo:
$$ \Norm{x+y+z}^2+\Norm{x-z}^2+\Norm{y-z}^2=\Norm{x+y-z}^2+\Norm{x+z}^2+\Norm{y+z}^2 $$
Ora moltiplico per due e applico l' identità del paralleogramma ai primi due addendi di ogni membro, ottenendo:
$$ \Norm{2x+y}^2+\Norm{y+2z}^2+2\Norm{y-z}^2=\Norm{2x+y}^2+\Norm{y-2z}^2+2\Norm{y+z}^2 $$
$$ \Leftrightarrow \Norm{y+2z}^2+2\Norm{y-z}^2=\Norm{y-2z}^2+2\Norm{y+z}^2 $$
da qui si conclude usando $(\star)$ per aprire le norme.
Manca adesso l' omogeneità. Si ha subito per omogeneità della norma che:
$$ \langle \lambda x|\lambda y\rangle= \lambda ^2 \langle x|y\rangle  \qquad \lambda\in \RR $$
usando induttivamente l' additività ottengo che:
$$ \Braket{\frac{a}{b}\, x|\frac{c}{d}\, y}=\frac{1}{b^2d^2}\left\langle ad\, x|cb\, y\right\rangle=\frac{ac}{bd}\left\langle x|y\right\rangle\qquad a,b,c,d \in \NN $$
questo prova l' omogeneità sui razionali, per continuità ciò è vero su $\RR$. 

\ex{6.39} Essendo $V$ a dimensione finita esiste, fissata una base, un isomorfismo $\psi$ con un certo $\RR^d$:
$$ \psi: \ V\ni\lambda_1\vec{e_1}+\ldots+\lambda_d\vec{e_d}\ \mapsto\  (\lambda_1,\ldots,\lambda_d)\in\RR^d $$
Pongo ora su $V$ la norma  {\it euclidea}:
$$ \Norm{v}_{\mathcal{E}}:=\max_i\{\norm{\lambda_i}\} $$
mostriamo che $B_{\mathcal{E}}$, la sfera unitaria, è compatta per successioni. Data una successione di vettori unitari in $V$ le loro componenti sono $d$ successioni di reali in $[-1,1]$ dunque per Bolzano-Weierstrass hanno una sottosuccessione convergente. Considero ora un' altra generica norma su $V$ e la penso come funzione dalla sfera unitaria in $\RR$:
$$\Norm{\cdot}\, : B_{\mathcal{E}}\rightarrow \RR$$
Questa funzione è continua rispetto alla metrica indotta da $\Norm{\cdot}_{\mathcal{E}}$, in effetti si ha che è lipschitziana:
$$ \Norm{u-v}\leq\sum_{i=1}^{d}\norm{u_i-v_i}\Norm{\vec{e_i}}\leq d\cdot\max_i\{\norm{u_i-v_i}\}\cdot \max_i\{\Norm{\vec{e_i}}\}=K \Norm{u-v}_{\mathcal{E}} $$
Quindi per Weierstrass ammette $\max=M$ e $\min=m$, che devono essere positivi. Dunque si ha, per un generico $u\in B_{\mathcal{E}}$: 
$$m\Norm{u}_{\mathcal{E}}\leq \Norm{u}_2\leq M\Norm{u}_{\mathcal{E}}$$
Abbiamo mostrato che ogni norma è equivalente a $\Norm{\cdot}_{\mathcal{E}}$; e poichè relazione di bi-lipschitz equivalenza gode della proprietà transitiva abbiamo finito.\\ 
Per la seconda domanda si prenda $V$ come l' insieme delle successioni a valori reali assolutamente sommabili, le seguenti norme non sono equivalenti:
$$ \Norm{x_n}_1=\sup_k \norm{x_k} \quad \mbox{      e       }\quad \Norm{x_n}_2=\sum_{k\in\NN} 2^{-k}\norm{x_k} $$
in effetti non inducono neppure la stessa topologia poichè esistono successioni di vettori di $V$ che rispetto alla prima non convergono mentre rispetto alla seconda si. (ad esempio $(x_n)_k=\delta_{n,k}$).

\ex{6.40} {\it Controesempio:} nello spazio $V$ delle successioni la cui somma dei valori assoluti converge considero il sottospazio $W$ delle successioni definitivamente nulle. Se prendo le successioni del tipo $x_n=2^{-n}\chi_{[0,k]}$ queste vivono in $W$ ma il loro limite (rispetto a $k$ si intende, cioè la successione $y_n=2^{-n}$) chiaramente no. 

\ex{6.41} {\it Completezza $\Rightarrow$ Convergenza assoluta.} Poichè la serie converge il termine generale è infinitesimo, per completezza di $\RR$ questo vuol dire che, posto $S_k=\sum_{i=0}^k x_i$, si ha:
$$ \epsilon>\sum_{i=M}^{N}\Norm{x_i}\geq \Norm{\sum_{i=M}^{N}x_i}=\Norm{S_N-S_M}\qquad\quad N,M \geq n_{\epsilon} $$
Dunque la successione $S_N\in V$ e di Cauchy e per completezza converge.\\ 
{\it Convergenza assoluta $\Rightarrow$ Completezza.} Supponiamo $V$ incompleto. Esiste dunque $(x_n)$ che ha la proprietà di Cauchy ma non converge in $V$. Usando nella definizione di Cauchy $\epsilon=2^{-j}$ estraggo una sottosuccessione $(x_{n(j)})$ tale che la successione di partenza stia definitivamente in $B_{2^{-j}}(x_{n(j)})$ e scelgo sempre $x_{n(j+1)}\in B_{2^{-j}}(x_{n(j)})$. Osservo che anche $(x_{n(j)})$ ha la proprietà di Cauchy e per $\rec{6.13}$ neanche lei converge. Posto $y_j:=x_{n(j+1)}-x_{n(j)}$ ho che:
$$ \sum_{j=1}^{\infty}\Norm{y_j}\leq \sum_{j\in\NN}2^{-j}=1\quad \Rightarrow\quad x_{n(k)}-x_{n(0)}=\sum_{i=1}^{k}y_j\rightarrow \ell\ \  \Rightarrow\ \  x_{n(k)}\mbox{ converge} $$ 
ma questo è assurdo.

\ex{6.50} {\it Numero di Lebesgue.} Sia dato un ricoprimento aperto $\{A_{\lambda}\}_{\lambda\in\Lambda}$. Supponiamo che l' $\inf$ dei raggi sia $0$. Questo significa che per ogni $\epsilon>0$ esiste una palla $B_{\epsilon}(x_{\epsilon})$ che non è contenuta interamente in nessun aperto del ricoprimento. Per ogni $n$, mi gioco questa proprietà con $\epsilon=1/n$ e definisco come $x_n$ il centro della palla $B_{\epsilon}(x_\epsilon)$. Questa successione ammette una sottosuccessione convergente $(x_{n_k})$ ad un certo elemento $x_{\infty}\in X$, ma questo punto limite deve stare in un certo $A_{\lambda}$ (perchè sono un ricoprimento); dunque esiste $r>0$ tale che $B_r(x_{\infty})\subseteq A_{\lambda}$. In particolare ci saranno infiniti punti della mia sottosuccessione nell' intorno $B_{r/2}(x_{\infty})$. Ma allora scelgo un elemento della sottosuccessione abbastanza avanti in modo che $n_k>2/r$, ottengo quindi che $$B_{1/{n_k}}(x_{n_k})\subseteq B_r(x_{\infty})\subseteq A_{\lambda}$$
ma questo è assurdo perchè viola la proprietà caratteristica di $x_{n_k}$ (ha la palla di raggio dato interamente contenuta in un aperto del ricoprimento).

\ex{6.51} (\NINI in spazi metrici) Scelgo una successione tale che $x_n \in F_n$. Tale successione vive in $F_1$ che è compatto dunque qui ammette una sottosuccessione convergente a $x_{\infty}$. Per ogni naturale $N$ osservo che la sottosuccessione vive definitivamente in $F_N$ e per compattezza qui vive il suo limite $x_{\infty}$, ciò prova che $x_{\infty}\in \bigcap_{n\in \NN}F_n$.

\ex{6.54} %IL teorema
\centerline{ {\it Compattezza per ricoprimenti} $\Rightarrow$ {\it Compattezza sequenziale}}
Supponiamo per assurdo che esista una successione $(x_n)_{n\in\NN}$ che non ammette sottosuccessioni convergenti. Questo significa che l' immagine di $(x_n)_{n\in\NN}$ non ha punti di accumulazione, cioè che:
$$ \forall x\in X\  \exists \epsilon_x > 0: \; \left( B_{\epsilon_x}(x)\setminus \{x\}\right) \cap (x_n)_{n\in\NN} = \emptyset $$
Chiaramente $\bigcup_{x\in X}B_{\epsilon_x}(x)\supseteq X$ dunque $\{B_{\epsilon_x}(x)\}_{x\in X}$ è un ricoprimento aperto; ma per ipotesi deve ammettere un sottoricoprimento finito che avrà centri $y_1,\ldots,y_k$. Per pigeonhole per almeno un $1\leq i\leq k$ ho che $B_{\epsilon_{y_i}}(y_i)$ contiene infiniti termini della successione, ma per costruzione ne può contenere al più uno. Assurdo.\\

\centerline{{\it Compattezza sequenziale} $\Rightarrow$ {\it Totale limitatezza}}
Fissato un $\epsilon > 0$ e scelto un $x_0 \in X$ definisco ricorsivamente una sequenza $(x_n)$ scegliendo ogni termine in modo che:
$$x_{n} \in X \setminus \bigcup_{i=0}^{n-1}B_\epsilon(x_i)$$
Opero una dicotomia ({\it cit.}):
\begin{itemize}
\item Se esiste $N$ tale che $X \setminus \bigcup_{i=0}^{N}B_\epsilon(x_i) = \emptyset $ ho trovato un ricoprimento finito;
\item Se un tale $N$ non esiste allora ho ottenuto una successione $(x_n)$. Per compattezza deve avere una sottosuccessione convergente (dunque di Cauchy): ma questo è assurdo perché comunque presi $m>n$ ho che:
$$ x_m\notin B_{\epsilon}(x_n)\ \Rightarrow\ d(x_m,x_n)>\epsilon $$
\end{itemize}

\centerline{{\it Compattezza sequenziale} $\Rightarrow$ {\it Completezza}}
Sia $(x_n)$ una successione di Cauchy; per compattezza ammette una sottosuccessione convergente ma, essendo di Cauchy $(x_n)$ converge per $\rec{6.13}$.\\

\centerline{{\it Completezza} $\wedge$ {\it Totale limitatezza} $\Rightarrow$ {\it Compattezza sequenziale}}
Prendo una qualsiasi successione $(x_n)\subseteq X$; voglio definire ricorsivamente una sua sottosuccessione convergente; per completezza mi basta farla di Cauchy. Per totale limitatezza posso ricoprire tutto $X$ con numero finito di palle di raggio arbitrariamente fissato (diciamo $r_0=1$). La mia successione dovrà cadere infinite volte in almeno una di queste palle (per pigeonhole); ne prendo una e la chiamo $B_0$ e, poichè un sottoinsieme di un totalmente limitato è totalmente limitato, anche $B_0$ è totalmente limitato. Ma $B_0$ contiene infiniti termini della successione quindi, dato un qualunque suo ricoprimento finito posso trovare una $B_1\subset B_0$ di raggio arbitrario ($r_1=1/2$) che contiene ancora infiniti elementi della successione. Definisco così induttivemente i $B_k$ e li scelgo ogni volta di raggio $2^{-k}$. Definisco ora la mia sottosuccessione $(x_{n_k})_{k\in\NN}$ così:
$$ n_{k}:=\min\left\{j\in\NN\mid j>n_{k-1}\, \wedge\, x_j\in B_k\right\} $$ 
perchè l' insieme di cui prendo il minimo non è mai vuoto. Ma la sottosuccessione così definita è chiaramente di Cauchy.\\

\centerline{{\it Compattezza sequenziale $\Rightarrow$ Compattezza per ricoprimenti}}
Sia dato un ricoprimento aperto $\{A_{\lambda}\}_{\lambda\in\Lambda}$. Per compattezza sequenziale esiste un {\it numero di Lebesgue $\rho$} positivo (\rec{6.50}). Per totale limitatezza esiste un ricoprimento finito con palle di raggio $\rho$; sia $\{B_{\rho}(x_1),\ldots,B_{\rho}(x_n)\}$ tale ricoprimento finito. Per definizione di numero di Lebesgue ho che ognuna di queste palle $B_{\rho}(x_i)$ è contenuta in almeno un aperto $A_{\lambda_i}$ del mio ricoprimento; si ha dunque che $\{A_{\lambda_1},\ldots,A_{\lambda_n}\}$ è ancora un ricoprimento del mio insieme ed è chiaramente finito.\\

\centerline{{\it Totale limitatezza} $\Rightarrow$ {\it Separabilità}}
Voglio costruire un sottoinsieme denso e numerabile. Per ogni $\epsilon_k=2^{-k}$ ho un ricoprimento finito di $\epsilon_k-$palle. Sia $C_k=\{c_1,\ldots,c_{n(k)}\}$ l' insieme dei centri di queste palle. L' insieme $C=\bigcup_{k\in\NN}C_k$ è numerabile (in quanto unione numerabile di insiemi finiti) e denso, infatti dato un qualsiasi $x\in X$ ed $\epsilon>0$ so che $x$ è contenuto in una palla del mio $\epsilon-$ricoprimento finito, cioè dista meno di $\epsilon$ dal un centro di queste palle, che è un elemento di $C$. Per l'arbitrarietà di $\epsilon$ ogni palla centrata in $x$ interseca $C$.\\

\ex{6.55}
\centerline{{\it Separabilità $\Rightarrow$ Topologia a base numerabile}}
Per ipotesi esiste $D$ un sottoinsieme denso e numerabile. Voglio mostrare che
$$\mathcal{B}=\{B_q(d)\mid q\in\QQ^+ \wedge d\in D\}$$
è una base per la mia topologia, essendo chiaramente $\norm{\mathcal{B}}=\norm{D\cart \QQ}=\aleph_0$. Prendo un generico $A\subseteq X$ aperto e un generico $a\in A$. So che esiste $r>0$ tale che $B_{2r}(a)\subseteq A$. Ma per densità posso scegliere un $b$ in $B_r(a)\cap D$; a questo punto scelgo un razionale $q$ tale che $d(a,b)<q<r$ ed ho che:
$$ \{a\}\subseteq B_q(b)\subseteq B_{2r}(a)\subseteq A $$
prendendo a primo e ultimo membro l' unione per ogni $a\in A$ ottengo che $A$ si scriveva come unione di palle di raggio razionale e centro in $D$ cioè come unione di elementi di $\mathcal{B}$. Osserviamo che in realtà l' unione che abbiamo fatto è {\it sovrabbondante}, in effetti si può fare numerabile perchè stiamo unendo elementi di una famiglia numerabile ($\mathcal{B}$).\\

\centerline{{\it Topologia a base numerabile $\Rightarrow$ Separabilità}}
Scegliamo un rappresentante per ogni insieme in $\mathcal{B}$. Mostriamo che quest' insieme è denso (è ovviamente numerabile essendolo $\mathcal{B}$), per farlo basta osservare che ogni palla aperta centrata in ogni punto è un aperto, dunque si scrive come unione di elementi di $\mathcal{B}$, dunque contiene un elemento che avevo scelto in partenza.\\

\ex{6.57} Supponiamo sia separabile, esiste dunque un insieme $D$ denso e numerabile, sia poi $\{A_{\lambda}\}_{\lambda\in\Lambda}$ la nostra famiglia di aperti disgiunti. Per densità si ha che per ogni $\lambda$ esiste $d_{\lambda}\in D$ tale che $d_{\lambda}\in D \cap A_{\lambda}$. Ho dunque una mappa da $\Lambda$ a $D$, si vede che è iniettiva per disgiunzione, da cui l' assurdo per cardinalità.

\ex{6.59} Posso fare delle funzioni continue che sugli interi valgono solo $\{0, 1\}$. Per ogni numero reale $\xi \in[0,1)$ considero la sua rappresentazione binaria, e gli associo la funzione $\varphi _{\xi}$ che nell' $n$ esimo intero vale l' $n-$esima cifra binaria di $\xi$. Ma le palle aperte di raggio $1/2$ centrate in tali funzioni sono digiunte (infatti la distaza tra due di queste diverse è sempre $1$), si conclude per \rec{6.57}.

\ex{6.60} {\it (Teorema di Lindelof).} Sia $\{A_{\lambda}\}_{\lambda\in\Lambda}$ il mio ricoprimento aperto e $D$ il mio denso e numerabile, per $\rec{6.55}$ ho $\mathcal{B}$ una base numerabile per la topologia. Prendo ora un generico $x\in X$ e osservo che appartiene ad un $A_{\lambda_x}$, ma quest' ultimo si scrive come unione {\it numerabile} di elementi di ${\mathcal{B}}$; dunque esiste $B_x \in \mathcal{B}$ tale che $x\in B_x\subseteq A_{\lambda_x}$. Resta dunque definita una funzione:
$$ \varphi:\,  X \rightarrow \mathcal{B} \rightarrow \Lambda $$
$$ \varphi: x\, \mapsto\, B_x\, \mapsto \, \lambda_x $$
Essendo i $B_x$ numerabili ho che $\varphi(X)$ è numerabile, ma praticamente per definizione $\displaystyle \bigcup_{\lambda \in \varphi (X)}A_{\lambda}\supseteq X$.\\

\ex{6.62} ({\it Lemma di Baire}) Passando al complementare, voglio mostrare che non è possibile ottenere l'insieme vuoto come intersezione numerabile di aperti densi (qualsiasi palla $B$ non è tutta contenuta in $X$, quindi esiste almeno un $y \in X^c \cap B$). Supponiamo per assurdo esista $\{A_n\}_{n\in\NN}$ collezione di aperti densi a intersezione vuota; costruiamo una successione $(x_n) \tc x_1 \in A_1 \wedge \exists \delta_1 > 0 \quad B_{\delta_1}(x_1) \subseteq A_1, x_2 \in A_2 \cap B_{\delta_1}(x_1) \wedge \exists \delta_2 > 0 \tc B_{\delta_2}(x_2) \subset B_{\delta_1}(x_1), \ldots $. La successione $(x_n)$ così costruita è di Cauchy (è dentro una palla di raggio $\delta_k \forall n \ge k$ e posso scegliere la successione del $\delta_k$ in modo che sia infinitesima e ogni $y$ nel bordo di $B_{\delta_{n+1}}(x_{n+1})$ sia un punto interno di $B_{\delta_n}(x_n)$ ). Quindi $x_n \rightarrow x \in \bigcap_{n\in\NN} A_n\neq \emptyset$, assurdo.

