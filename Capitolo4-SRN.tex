\section{Serie Numeriche}

\ex{4.5} Usare la formula per $\tan(\alpha - \beta)$ e telescopizzzare.

\ex{4.13} Moltiplicate per 3 $\ldots$ \hide{\`e il cubo di un binomio telescopico}

\ex{4.17} Usate la sommazione per parti di Abel, ovvero \rec{1.25}. (Non \`e affatto inutile come sembra)

\ex{4.19} Usate nuovamente la sommazione per parti di Abel (\rec{1.25})

\ex{4.21} Ancora sommazione per parti (\rec{1.25})

\ex{4.22} Pongo $\frac{b_n}{a_n} = 1 + \epsilon_n$: serve che $\sum (-1)^n a_n$ converga e $\sum (-1)^n \epsilon_n a_n$ diverga. Trovare tali $a_n$ e $\epsilon_n$.

\ex{4.27} Osservare che $\frac{1}{n-1} = \sum_{j \ge 1} \frac{1}{n^j}$ e fattorizzare.

\ex{4.32} Dare la formula chiusa per $\sum_{k=1}^{n} \sin(k)$ scrivendola come geometrica di esponenziali complessi. Poi usare sommazione per parti di Abel (\rec{1.25}).
