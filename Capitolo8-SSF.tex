\section{Successioni e Serie di Funzioni}
\ex{8.15} Sia $(X,d)$ uno spazio metrico completo, $(Y, \delta)$ uno spazio metrico qualsiasi, e 
$f= \lim_{n \to infty} f_n$ una funzione da $X$ in $Y$ limite puntuale di funzioni. Allora le discontinuità di $f$ sono 
un unione numerabile di chiusi a parte interna vuota.
\textbf{Caratterizzazione delle discontinuità}. Intuitivamente $f$ è discontinua in $x$ se le $x$ vicine 'tardano ad arrivare' al loro valore limite. Formalmente, negando la convergenza uniforme in ogni palletta di raggio $1/n$ centrata in $x$ si può dimostrare che $x \in \mathrm{disc}(f)$ sse esiste un $\varepsilon(x) > 0$ e una successione $x_n \rightarrow x$ tale che $\delta( f_n(x_n), f(x_n) ) \ge \epsilon$ per ogni $n \in \mathbb{N}$. Via perciò con l'esaustione: detto $E_k := \{x \in \mathrm{disc}(f): \varepsilon(x) \ge 1/k\}$, abbiamo 
$$ \mathrm{disc}(f) = \bigcup_{k \in \mathbb{N}} E_k $$
Se gli $E_k$ fossero chiusi e a parte interna vuota, allora $\mathrm{open} (\mathrm{disc}(f)) = \emptyset$ , 
da cui $\mathrm{close}( \mathrm{disc}(f)^{c} ) = \emptyset^c = X$, ossia $f$ sarebbe continua su un denso.

\textbf{$E_k$ è chiuso.} 
Se $x_n \rightarrow x$, $x_n \in E_k$, allora esiste $\forall n$ una successione \newline 
$(z^n_m )_{m \in \mathbb{N}} \rightarrow x_n$ tale che $\delta(f(z^n_m), f_m(z^n_m) ) \ge 1/k$. \newline
Ma allora $(z^n_n) \rightarrow x$ e $\delta(f(z^n_n), f_n(z^n_n) ) \ge 1/k$. \newline

\textbf{$E_k$ è a parte interna vuota.} 
Supponiamo esista un aperto $B_{\varepsilon}(x_0) \subseteq E_k$. In particolare c'è anche la palla chiusa $ J := \mathrm{close}(B_{\varepsilon/2}(x_0) ) $, che è anche completa perchè sottospazio chiuso di $X$ completo. \newline 
Per ogni $x \in J$, sia $r(x) := \min \{r \in \mathbb{N}: \forall m,n \ge r \ \ \delta(f_m(x), f_n(x) ) \le 1/2k\} $. Esiste perchè $(f_n(x))$ è convergente, perciò di Cauchy. \newline
Siano adesso $C_s = \{x \in J: r(x) \le s\}$ per ogni $s \in \mathbb{N} $. \newline
In pratica stiamo esaurendo i punti di $J$ in base a 'quanto ci mettono' per arrivare abbastanza vicini al loro valore limite. Dimostriamo che i $C_s$ sono chiusi. Sia $(x_i) \rightarrow x $ una successione in $C_s$. Allora, per ogni $m,n \ge s $ abbiamo 
$$ \delta( f_m(x_i), f_n(x_i) ) \ge 1/2k $$
da cui passando al limite in $i$ si ha $\delta( f_m(x), f_n(x) ) \ge 1/2k $ per ogni $m,n \ge s$, e dunque $r(x) \le s \ \Rightarrow x \in C_s$. Visto che
$$\bigcup_{s \in \mathbb{N}} C_s = J, \ J \text{ completo} $$
per il lemma di Baire esiste $r\in \mathbb{N}$ tale che $C_r$ contiene una palletta aperta $B_{\rho}(y)$. Eh, ma adesso in $B_{\rho}$ si sta vicini al valore limite, e in $E_k$ c'è qualcuno che sta lontano! Formalmente, abbiamo
$$ (1) \ \ y \in J \subseteq E_k \ \Rightarrow \exists (z_n) \rightarrow y: \delta( f_n(z_n), f(z_n) ) \ge 1/k  $$
$$ (2) \ \ \text{definitivamente } z_n \in B_{\rho}(y) \subseteq C_r \Rightarrow \forall i,j \ge r  \ \ \ \delta( f_i(z_n), f_j(z_n) ) \le 1/2k $$
ma per $n \ge r$, usando la (2) con $i=n$, $j \to \infty$ otteniamo \newline $\delta( f_n(z_n), f(z_n) ) \le 1/2k$, in contraddizione con la (1).
\newline
\textbf{$E_k$ è a parte interna vuota.} 
\ex{8.46}{\it Teorema della mappa aperta.} Siano $X$ e $Y$ due Banach, $T:X\rightarrow Y$ mappa lineare, continua e bigettiva. Vogliamo mostrare che $T$ manda aperti in aperti. Per linearità basta mostrare che manda palle centrate nell' origine in aperti.\\
Passo $(1)$: {\it Esiste} $c>0$ {\it tale che} $\overline{T(B_X (0,1))}\supseteq B_Y (0, 2c)$.\\
Per surgettività si ha:
$$
Y=\bigcup_{k\in\NN} {\overline{T(B_X (0,k))}}
$$
ma $Y$ è completo, dunque, per il Lemma di Baire, almeno un elemento dell' unione ha parte interna nonvuota. Esistono dunque $r>0$, $y_o\in Y$ e $n\in\NN$ tale che:
$$
\overline{T(B_X (0,n))}\supseteq B_Y(y_o,r) 
$$
trasliamo ora questo contenimento nell' origine; posto $x_o:=T^{-1}(y_o)$ abbiamo, per ogni $y$ di norma più piccola di $r$:
$$
y_o+y=\lim_{k\rightarrow \infty} T(x_k)=T(x_o)+ \lim_{k\rightarrow \infty} T(x_k-x_o)
$$
cioè:
$$
y=\lim_{k\rightarrow \infty} T(x_k-x_o)\quad \Rightarrow\quad B_Y(0,r)\subseteq \overline{T(B_X(0,2n))}
$$
infatti $\Norm{x_k-x_o}_X\leq 2n$. Per omogeneità possiamo omotetizzare questo contenimento ottenendo la tesi con $c=r/4n$.\\
\\
Passo $(2)$: {\it Vale in realtà} $T(B_X(0,1))\supseteq B_Y(0,c)$.\\
Prendiamo un generico $y\in B_Y(0,c)\subseteq \overline{T(B_X(0,1/2))}$ per definizione posso trovare $y_1=T(x_1)$ più vicino di $c/2$ ad $y$, ossia tale che valgano le seguenti:
$$
\Norm{y-y_1}_Y<\frac{c}{2}\quad \mbox{e}\quad \Norm{x_1}_X<\frac{1}{2}
$$
Ora osservo che intanto:
$$
B_Y(0,c/2)\subseteq \overline{T(B_X(0,1/4))}
$$
inoltre il vettore $(y-y_1)\in B_Y(0,c/2)$, dunque come prima trovo $y_2=T(x_2)$ tale che:
$$
\Norm{y-y_1-y_2}_Y<\frac{c}{4}\quad \mbox{e}\quad \Norm{x_2}_X<\frac{1}{4}
$$
Continuando a ragionare così trovo due successioni $(y_k)\in Y$ e $(x_k)\in X$ tali che per ogni $k$ valga $y_k=T(x_k)$ e che valgano, per ogni $m$, le due seguenti:
$$
\Norm{y-\sum_{i=1}^m y_{i}}_Y<\frac{c}{2^m}\quad \mbox{e}\quad \Norm{x_m}_X<\frac{1}{2^m}\qquad\qquad (\star)
$$
Ora osservo che la serie $\sum_{m\in\NN}\Norm{x_m}_X$ converge dunque per completezza di $X$ ho che esiste $x$ tale che:
$$
x=\sum_{m=1}^{\infty} x_m\qquad \mbox{inoltre} \qquad \Norm{x}_X\leq\sum_{m=1}^{\infty} \Norm{x_m}_X<1\quad \Rightarrow x\in B_X(0,1)
$$
inoltre la prima equazione di $(\star)$ ci dice (per definizione) che $y=\sum_{i=1}^{\infty}y_i$. Usando la continuità di $T$ abbiamo finito:
$$
y=\sum_{i=1}^{\infty}y_i=\sum_{i=1}^{\infty}T(x_i)=T\left(\sum_{i=1}^{\infty}x_i\right)=T(x)
$$
cioè $y\in T(B_X(0,1))$.\\
Applicando alla tesi del passo $(2)$ l' operatore $T^{-1}$ e usando l'omegenità si ha:
$$
T^{-1}(B_Y(0,1))\subseteq B_X(0,1/c)
$$
Ossia $T^{-1}$ è limitato, dunque continuo.
