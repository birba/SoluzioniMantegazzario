\section{Successioni e Serie di Funzioni}
\ex{8.12} $K\subseteq\mathbb R$ compatto e $f_n$ continue su $K$, convergenti puntualmente ad $f$ continua, $f_n\ge f_{n+1}$. Allora la convergenza è uniforme.

Supponiamo che non sia uniforme: allora esistono un $h>0$ ed una successione di funzioni e di punti, siano $f_i$ ed $x_i$ rispettivamente ($i\in I\subseteq\mathbb N$ con $I$ infinito), tali che $f_i(x_i)-f(x_i)>h$. Essendo $K$ compatto, esiste una sottosuccessione $x_j$ ($j\in J\subseteq I$) convergente ad $x_\infty$. Per ipotesi vale $f_j(x_\infty)-f(x_\infty)\rightarrow 0$, quindi esiste $j_*\in J$ tale che $f_{j_*}(x_\infty)-f(x_\infty)<\frac{h}{4}$; inoltre, essendo $f_{j_*}$ e $f$ continue in $x_\infty$, esiste $\delta$ tale che $[\abs{x-x_\infty}<\delta\Rightarrow\abs{f_{j_*}(x)-f_{j_*}(x_\infty)}<\frac{h}{4} \quad \abs{f(x)-f(x_\infty)}<\frac{h}{4}]$; dal momento che $x_j\rightarrow x_\infty$ esiste $\bar j\ge j_*$ tale che $\abs{x_{\bar j}-x_\infty}<\delta$. Infine, mettendo tutto assieme vale: $h<f_{\bar j}(x_{\bar j})-f(x_{\bar j})\le f_{j_*}(x_{\bar j})-f(x_{\bar j})<(f_{j_*}(x_\infty)+\frac{h}{4})-(f(x_\infty)-\frac{h}{4})=f_{j^*}(x_\infty)-f(x_\infty)+\frac{h}{2}<\frac{3h}{4}$, assurdo.


\ex{8.15} Sia $(X,d)$ uno spazio metrico completo, $(Y, \delta)$ uno spazio metrico qualsiasi, e 
$f= \lim_{n \to infty} f_n$ una funzione da $X$ in $Y$ limite puntuale di funzioni. Allora le discontinuità di $f$ sono un unione numerabile di chiusi a parte interna vuota.
\textbf{Caratterizzazione delle discontinuità}. Intuitivamente $f$ è discontinua in $x$ se le $x$ vicine 'tardano ad arrivare' al loro valore limite. Formalmente, negando la convergenza uniforme in ogni palletta di raggio $1/n$ centrata in $x$ si può dimostrare che $x \in \mathrm{disc}(f)$ sse esiste un $\varepsilon(x) > 0$ e una successione $x_n \rightarrow x$ tale che $\delta( f_n(x_n), f(x_n) ) \ge \epsilon$ per ogni $n \in \mathbb{N}$. Via perciò con l'esaustione: detto $E_k := \{x \in \mathrm{disc}(f): \varepsilon(x) \ge 1/k\}$, abbiamo 
$$ \mathrm{disc}(f) = \bigcup_{k \in \mathbb{N}} E_k $$
Se gli $E_k$ fossero chiusi e a parte interna vuota, allora $\mathrm{open} (\mathrm{disc}(f)) = \emptyset$ , 
da cui $\mathrm{close}( \mathrm{disc}(f)^{c} ) = \emptyset^c = X$, ossia $f$ sarebbe continua su un denso.

\textbf{$E_k$ è chiuso.} 
Se $x_n \rightarrow x$, $x_n \in E_k$, allora esiste $\forall n$ una successione \newline 
$(z^n_m )_{m \in \mathbb{N}} \rightarrow x_n$ tale che $\delta(f(z^n_m), f_m(z^n_m) ) \ge 1/k$. \newline
Ma allora $(z^n_n) \rightarrow x$ e $\delta(f(z^n_n), f_n(z^n_n) ) \ge 1/k$. \newline

\textbf{$E_k$ è a parte interna vuota.} 
Supponiamo esista un aperto $B_{\varepsilon}(x_0) \subseteq E_k$. In particolare c'è anche la palla chiusa $ J := \mathrm{close}(B_{\varepsilon/2}(x_0) ) $, che è anche completa perchè sottospazio chiuso di $X$ completo. \newline 
Per ogni $x \in J$, sia $r(x) := \min \{r \in \mathbb{N}: \forall m,n \ge r \ \ \delta(f_m(x), f_n(x) ) \le 1/2k\} $. Esiste perchè $(f_n(x))$ è convergente, perciò di Cauchy. \newline
Siano adesso $C_s = \{x \in J: r(x) \le s\}$ per ogni $s \in \mathbb{N} $. \newline
In pratica stiamo esaurendo i punti di $J$ in base a 'quanto ci mettono' per arrivare abbastanza vicini al loro valore limite. Dimostriamo che i $C_s$ sono chiusi. Sia $(x_i) \rightarrow x $ una successione in $C_s$. Allora, per ogni $m,n \ge s $ abbiamo 
$$ \delta( f_m(x_i), f_n(x_i) ) \ge 1/2k $$
da cui passando al limite in $i$ si ha $\delta( f_m(x), f_n(x) ) \ge 1/2k $ per ogni $m,n \ge s$, e dunque $r(x) \le s \ \Rightarrow x \in C_s$. Visto che
$$\bigcup_{s \in \mathbb{N}} C_s = J, \ J \text{ completo} $$
per il lemma di Baire esiste $r\in \mathbb{N}$ tale che $C_r$ contiene una palletta aperta $B_{\rho}(y)$. Eh, ma adesso in $B_{\rho}$ si sta vicini al valore limite, e in $E_k$ c'è qualcuno che sta lontano! Formalmente, abbiamo
$$ (1) \ \ y \in J \subseteq E_k \ \Rightarrow \exists (z_n) \rightarrow y: \delta( f_n(z_n), f(z_n) ) \ge 1/k  $$
$$ (2) \ \ \text{definitivamente } z_n \in B_{\rho}(y) \subseteq C_r \Rightarrow \forall i,j \ge r  \ \ \ \delta( f_i(z_n), f_j(z_n) ) \le 1/2k $$
ma per $n \ge r$, usando la (2) con $i=n$, $j \to \infty$ otteniamo \newline $\delta( f_n(z_n), f(z_n) ) \le 1/2k$, in contraddizione con la (1).
\newline
\textbf{$E_k$ è a parte interna vuota.} 

\ex{8.15 \notes{Seconda Soluzione}} $f_n:\mathbb R\rightarrow\mathbb R$ continue convergono puntualmente ad $f$. Allora l'insieme dei punti di discontinuità di $f$ è di prima categoria.

Voglio scrivere l'insieme dei punti di discontinuità di $f$ come unione numerabile di chiusi a parte interna vuota. Considero al variare di $n\in\mathbb N$ gli insiemi $O_f(\frac{1}{n}):=\{x\in\mathbb R : \theta_f(x)\ge\frac{1}{n}\}$: ovviamente l'unione di tale famiglia numerabile di insiemi è l'insieme dei punti di discontinuità di $f$. Dimostro ora che sono chiusi e a parte interna vuota.

$O_f(\frac{1}{n})$ è chiuso: presa una successione di punti $\langle x_i\in O_f(\frac{1}{n}) | i\in \mathbb N\rangle$ tali che $x_i\rightarrow x_\infty$, voglio dimostrare che $\theta_f(x_\infty)\ge\frac{1}{n}$: dato che $x_i\rightarrow x_\infty$ ho che $\forall\epsilon\exists k_\epsilon [\abs{x_{k_\epsilon}-x_\infty}<\epsilon]$; inoltre, dato che $\theta_f(x_i)\ge\frac{1}{n}$, ho che $\forall i\forall\epsilon\forall\delta\exists a(i,\epsilon,\delta)b(i,\epsilon,\delta)$ tali che $\abs{a(i,\epsilon,\delta)-x_i}<\epsilon \quad \abs{b(i,\epsilon,\delta)-x_i}<\epsilon$ e $\abs{f(a(i,\epsilon,\delta))-f(b(i,\epsilon,\delta))}>\frac{1}{n}-\delta$. Ma allora ho che $\abs{a(k_\frac{\epsilon}{2},\frac{\epsilon}{2},\delta)-x_\infty}\le\abs{a(k_\frac{\epsilon}{2},\frac{\epsilon}{2},\delta)-x_{k_{\frac{\epsilon}{2}}}}+\abs{x_{k_{\frac{\epsilon}{2}}}-x_\infty}<\frac{\epsilon}{2}+\frac{\epsilon}{2}=\epsilon$ (e disuguaglianza analoga per la $b$) da cui si ha che $\forall\epsilon\forall\delta\exists a(k_\frac{\epsilon}{2},\frac{\epsilon}{2},\delta) b(k_\frac{\epsilon}{2},\frac{\epsilon}{2},\delta)$ tali che $\abs{f(a(k_\frac{\epsilon}{2},\frac{\epsilon}{2},\delta))-f(b(k_\frac{\epsilon}{2},\frac{\epsilon}{2},\delta))}>\frac{1}{n}-\delta$ che per definizione vuol dire $\theta_f(x_\infty)\ge\frac{1}{n}$.

$O_f(\frac{1}{n})$ è a parte interna vuota: supponiamo che ci sia un intervallo chiuso $I$ completamente contenuto in $O_f(\frac{1}{n})$: all'interno di quell'intervallo definisco gli insiemi $E(m_*,\lambda):=\{x\in I : \forall m\ge m_*[\abs{f_m(x)-f(x)}\le\lambda]\}$. Dimostro ora che la chiusura di $E(m_*,\frac{1}{6n})$ ha parte interna vuota $\forall m_*$.

$\overline{E}(m_*,\frac{1}{6n})\subseteq E(m_*,\frac{1}{3n})$: infatti se $x_i\in E(m_*,\frac{1}{6n})$ è una successione di punti che convergono ad $x_\infty$, allora, per ogni $p\ge m_*$ e per ogni $\mu$, per continuità di $f_p$ ho un indice $i(p,\mu)$ tale che $\forall i\ge i(p,\mu) \quad \abs{f_p(x_i)-f_p(x_\infty)}<\mu$; quindi, fissato un arbitrario $m\ge m_*$ (ed essendo $x_i\in E(m_*,\frac{1}{6n})$) vale $\abs{f_m(x_\infty)-f(x_\infty)}=\lim_{p\rightarrow\infty}\abs{f_m(x_\infty)-f_p(x_\infty)}$ ma, detto $\phi:=\max\{i(m,\mu),i(p,\mu)\}$, vale $\abs{f_m(x_\infty)-f_p(x_\infty)}\le\abs{f_m(x_\infty)-f_m(x_\phi)}+\abs{f_m(x_\phi)-f_p(x_\phi)}+\abs{f_p(x_\phi)-f_p(x_\infty)}<\mu+\abs{f_m(x_\phi)-f_p(x_\phi)}+\mu\le2\mu+\abs{f_m(x_\phi)-f(x_\phi)}+\abs{f(x_\phi)-f_p(x_\phi)}\le2\mu+2\frac{1}{6n}=2\mu+\frac{1}{3n}$ quindi $\lim_{p\rightarrow\infty}\abs{f_m(x_\infty)-f_p(x_\infty)}\le\frac{1}{3n}$ quindi $x_\infty\in E(m_*,\frac{1}{3n})$: basta dunque dimostrare che $E(m_*,\frac{1}{3n})$ ha parte interna vuota. Supponiamo per assurdo che $E(m_*,\frac{1}{3n})$ abbia parte interna (diciamo un intervallo $J\subseteq I$): allora, scelto $l\in J$, essendo $\theta_f(l)\ge\frac{1}{n}$ (perchè siamo in $I$), esistono in $J$ due successioni (eventualmente costanti uguali ad $l$) $x_j$ e $z_j$ convergenti ad $l$ e tali che $\lim_{j\rightarrow\infty}\abs{f(x_j)-f(z_j)}\ge\frac{1}{n}$. Inoltre $\abs{f_{m_*}(x_j)-f_{m_*}(z_j)}\le\abs{f_{m_*}(x_j)-f_{m_*}(l)}+\abs{f_{m_*}(l)-f_{m_*}(z_j)}$ quindi per i carabinieri (e per continuità di $f_{m_*}$) $\lim_{j\rightarrow\infty}\abs{f_{m_*}(x_j)-f_{m_*}(z_j)}=0$. Ma (dato che siamo in $J$) $\abs{f(x_j)-f(z_j)}\le\abs{f(x_j)-f_{m_*}(x_j)}+\abs{f_{m_*}(x_j)-f_{m_*}(z_j)}+\abs{f_{m_*}(z_j)-f(z_j)}\le\frac{1}{3n}+\abs{f_{m_*}(x_j)-f_{m_*}(z_j)}+\frac{1}{3n}$ e quindi ancora per i carabinieri $\frac{1}{n}\le\lim_{j\rightarrow\infty}\abs{f(x_j)-f(z_j)}\le\frac{2}{3n}+\lim_{j\rightarrow\infty}\abs{f_{m_*}(x_j)-f_{m_*}(z_j)}=\frac{2}{3n}$ assurdo.
Quindi $\overline{E}(m_*,\frac{1}{6n})$ ha parte interna vuota; ma al variare di $m_*\in\mathbb N$ $\overline{E}(m_*,\frac{1}{6n})$ ricoprono $I$: avrei dunque che $I$ si scrive come unione numerabile di chiusi a aprte interna vuota, assurdo per Baire.
Quindi $O_f(\frac{1}{n})$ è a parte interna vuota.

Quindi la tesi è dimostrata.

\ex{8.26} $K$ spazio metrico compatto, $S\subseteq C(K)$. Allora
\begin{center}
$S$ compatto $\Leftrightarrow[S$ chiuso, $f\in S$ equicontinue, $f\in S$ equilimitate$]$
\end{center}

$(\Rightarrow)$ Se $S$ è compatto allora ovviamente $S$ chiuso.

Le funzioni di $S$ sono equicontinue: infatti fissato un punto $\overline x\in K$ ed un $\epsilon>0$, ricopro $S$ con la seguente famiglia di aperti: $A_n:=\{f\in S : [d(x,\overline x)\le\frac{1}{n}\Rightarrow\abs{f(x)-f(\overline x)}<\epsilon]\}$ (è abbastanza semplice verificare che $A_n$ è aperto); questo è un ricoprimento in quanto $S\subseteq C(K)$, ed essendo $S$ compatto ammette un sottoricoprimento finito, da cui $\forall x\forall\epsilon\exists\frac{1}{n}\forall f\in S[d(x,\overline x)<\frac{1}{n}\Rightarrow\abs{f(x)-f(\overline x)}<\epsilon]$, che è la definizione di equicontinuità.

Le funzioni di $S$ sono equilimitate: infatti ricopro $S$ con la seguente famiglia di aperti: $A_n:=\{f\in S : \norma{f}<n\}$ (anche in questo caso la verifica che $A_n$ è aperto è semplice); essendo $S$ compatto, esiste un sottoricoprimento finito, cioè un $n$ che limita contemporaneamente tutte le $F$, che è la definizione di equilimitatezza.

$(\Leftarrow)$ Supponiamo $S$ chiuso e contenente funzioni equicontinue ed equilimitate. Consideriamo una successione di funzioni $f_n\in S$ e vogliamo trovare un sottosuccessione convergente (dalla chiusura di $S$ avremo che il limite appartiene ad $S$). Essendo $K$ compatto, è separabile (siano $x_m$ densi in $K$): inoltre, per ogni punto $x\in K$ e per ogni successione di funzioni in $S$, essendo queste equilimitate, la successione dei valori delle funzioni in $x$ è di Cauchy, quindi c'è una sottosuccessione di funzioni convergenti nel punto $x$. Ora costruisco una sottosuccessione $T$ delle $f_n$ convergente su tutti i punti $x_m$: Prendo la prima delle $f_n$ e la uso come prima funzione della successione $T$ che voglio costruire, poi scelgo tra le restanti una sottosuccessione convergente nel punto $x_1$ e prendo la prima funzione come seconda funzione della successione $T$, poi scelgo una sotto-sottosuccessione convergente nel punto $x_2$ e uso la prima di tali funzioni come terza funzione della successione $T$, e così via. La sottosuccessione $f_i\in T$ che ottengo converge in tutti i punti $x_m$ del denso e numerabile ad una funzione $f$ (che per ora è definita sono sugli $x_m$). Dimostro ora che la convergenza avviene puntualmente anche negli altri punti e che è uniforme.

Preso un punto $\overline x$ ed un $\epsilon>0$, essendo le funzioni $f_i\in T$ equicontinue in $\overline x$ esiste $\delta(\epsilon,\overline x)$ tale che $\forall i[\norma{x-\overline x}<\delta(\epsilon,\overline x)\Rightarrow\abs{f_i(x)-f_i(\overline x)}<\epsilon]$; essendo $x_m$ densi, per ogni $\delta$ esiste un $m(\delta)$ tale che $\norma{x_{m(\delta)}-\overline x}<\delta$; inoltre per ogni $m$ e per ogni $\epsilon$, essendo le funzioni $f_i$ convergenti in $x_m$, esiste un indice $i(m,\epsilon)$ tale che $\forall j>i(m,\epsilon)$ vale $\abs{f_j(x_m)-f(x_m)}<\epsilon$. Quindi, scelti $\epsilon$ ed $\overline x$, si ha $\forall j,k>i(m(\delta(\frac{\epsilon}{3},\overline x)),\frac{\epsilon}{3})$ vale $\abs{f_j(\overline x)-f_k(\overline x)}\le\abs{f_j(\overline x)-f_j(x_{m(\delta(\frac{\epsilon}{3},\overline x))})}+\abs{f_j(x_{m(\delta(\frac{\epsilon}{3},\overline x))})-f_k(x_{m(\delta(\frac{\epsilon}{3},\overline x))})}+\abs{f_k(x_{m(\delta(\frac{\epsilon}{3},\overline x))})-f_k(\overline x)}<\frac{\epsilon}{3}+\frac{\epsilon}{3}+\frac{\epsilon}{3}=\epsilon$. Quindi la successione $f_i(\overline x)$ è di Cauchy, quindi converge, e posso definire $f$ su tutto $K$.

Mostro ora che $f$ così definita è continua. Preso $\overline x$, sia $\delta(\epsilon)$ il modulo di continuità comune delle $f_i$ in $\overline x$. Allora $\forall i[\norma{x-\overline x}<\delta(\epsilon)\Rightarrow\abs{f_i(x)-f_i(\overline x)}<\epsilon]$ e passando al limite in $i$ in questa disuguaglianza si ottiene $[\norma{x-\overline x}<\delta(\epsilon)\Rightarrow\abs{f(x)-f(\overline x)}<\epsilon]$ quindi $f$ è continua con lo stesso modulo di continuità delle $f_i$.

Infine mostro che la convergenza è uniforme: per ogni $\delta$ esiste un $m(\delta)$ tale che ogni punto del compatto dista meno di $\delta$ da almeno uno dei punti $x_1,...,x_{m(\delta)}$; inoltre, fissati $m$ ed $\epsilon$, essendo tali punti in numero finito ed essendoci convergenza puntuale in quei punti, esiste $i(m,\epsilon)$ tale che $\forall j\ge i\forall l\le m[\abs{f_j(x_l)-f(x_l)}<\epsilon]$ (notare che queso non è lo stesso $i(m,\epsilon)$ di prima, ma è il massimo tra gli $i(l,\epsilon)$ di prima al variare di $l\le m$). Ora, chamando $\delta(\epsilon)$ il modulo di continuità comune della $f_i$ e di $f$, vale: preso un $\epsilon$, per ogni $x$ vale $\exists l\le m(\delta(\frac{\epsilon}{3}))\quad\norma{x-x_l}<\delta(\frac{\epsilon}{3})$ da cui $\forall j\ge i(m(\delta(\frac{\epsilon}{3})),\frac{\epsilon}{3})$ vale $\abs{f_j(x)-f(x)}\le\abs{f_j(x)-f_j(x_l)}+\abs{f_j(x_l)-f(x_l)}+\abs{f(x_l)-f(x)}<\frac{\epsilon}{3}+\frac{\epsilon}{3}+\frac{\epsilon}{3}=\epsilon$ e dunque la convergenza è uniforme.


\ex{8.46}{\it Teorema della mappa aperta.} Siano $X$ e $Y$ due Banach, $T:X\rightarrow Y$ mappa lineare, continua e bigettiva. Vogliamo mostrare che $T$ manda aperti in aperti. Per linearità basta mostrare che manda palle centrate nell' origine in aperti.\\
Passo $(1)$: {\it Esiste} $c>0$ {\it tale che} $\overline{T(B_X (0,1))} \subseteq B_Y (0, 2c)$.\\
Per surgettività si ha:
$$
Y=\bigcup_{k\in\NN} {\overline{T(B_X (0,k))}}
$$
ma $Y$ è completo, dunque, per il Lemma di Baire, almeno un elemento dell' unione ha parte interna nonvuota. Esistono dunque $r>0$, $y_o\in Y$ e $n\in\NN$ tale che:
$$
\overline{T(B_X (0,n))}\supseteq B_Y(y_o,r) 
$$
trasliamo ora questo contenimento nell' origine; posto $x_o:=T^{-1}(y_o)$ abbiamo, per ogni $y$ di norma più piccola di $r$:
$$
y_o+y=\lim_{k\rightarrow \infty} T(x_k)=T(x_o)+ \lim_{k\rightarrow \infty} T(x_k-x_o)
$$
cioè:
$$
y=\lim_{k\rightarrow \infty} T(x_k-x_o)\quad \Rightarrow\quad B_Y(0,r)\subseteq \overline{T(B_X(0,2n))}
$$
infatti $\Norm{x_k-x_o}_X\leq 2n$. Per omogeneità possiamo omotetizzare questo contenimento ottenendo la tesi con $c=r/4n$.\\
\\
Passo $(2)$: {\it Vale in realtà} $T(B_X(0,1))\supseteq B_Y(0,c)$.\\
Prendiamo un generico $y\in B_Y(0,c)\subseteq \overline{T(B_X(0,1/2))}$ per definizione posso trovare $y_1=T(x_1)$ più vicino di $c/2$ ad $y$, ossia tale che valgano le seguenti:
$$
\Norm{y-y_1}_Y<\frac{c}{2}\quad \mbox{e}\quad \Norm{x_1}_X<\frac{1}{2}
$$
Ora osservo che intanto:
$$
B_Y(0,c/2)\subseteq \overline{T(B_X(0,1/4))}
$$
inoltre il vettore $(y-y_1)\in B_Y(0,c/2)$, dunque come prima trovo $y_2=T(x_2)$ tale che:
$$
\Norm{y-y_1-y_2}_Y<\frac{c}{4}\quad \mbox{e}\quad \Norm{x_2}_X<\frac{1}{4}
$$
Continuando a ragionare così trovo due successioni $(y_k)\in Y$ e $(x_k)\in X$ tali che per ogni $k$ valga $y_k=T(x_k)$ e che valgano, per ogni $m$, le due seguenti:
$$
\Norm{y-\sum_{i=1}^m y_{i}}_Y<\frac{c}{2^m}\quad \mbox{e}\quad \Norm{x_m}_X<\frac{1}{2^m}\qquad\qquad (\star)
$$
Ora osservo che la serie $\sum_{m\in\NN}\Norm{x_m}_X$ converge dunque per completezza di $X$ ho che esiste $x$ tale che:
$$
x=\sum_{m=1}^{\infty} x_m\qquad \mbox{inoltre} \qquad \Norm{x}_X\leq\sum_{m=1}^{\infty} \Norm{x_m}_X<1\quad \Rightarrow x\in B_X(0,1)
$$
inoltre la prima equazione di $(\star)$ ci dice (per definizione) che $y=\sum_{i=1}^{\infty}y_i$. Usando la continuità di $T$ abbiamo finito:
$$
y=\sum_{i=1}^{\infty}y_i=\sum_{i=1}^{\infty}T(x_i)=T\left(\sum_{i=1}^{\infty}x_i\right)=T(x)
$$
cioè $y\in T(B_X(0,1))$.\\
Applicando alla tesi del passo $(2)$ l' operatore $T^{-1}$ e usando l'omegenità si ha:
$$
T^{-1}(B_Y(0,1))\subseteq B_X(0,1/c)
$$
Ossia $T^{-1}$ è limitato, dunque continuo.
