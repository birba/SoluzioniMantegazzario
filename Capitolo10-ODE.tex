\section{Equazioni Differenziali Ordinarie}
\ex{10.5} {\it Unicità.} Immediata.\\
{\it Esistenza.} Considero la funzione $h$ da $K$ a $\RR$ definita da $h(x)=d(x,f(x))$; è facile vedere che $h$ è continua (usando la continuità di $f$) su $K$ e dunque ammetterà minimo $m=h(z)$ per un certo $z\in K$. Se $m=0$ avremmo finito per nondegenericità della distanza, supponiamo dunque $m>0$. Si ha subito un assurdo:
$$
m=h(z)=d(z,f(z))>d(f(z),f^2(z))=h(f(z))\quad \Rightarrow\quad\mbox{assurdo perchè $z$ era un minimo per $h$} 
$$ 
Mostriamo ora che tale $z$ è il limite delle ricorrenze del tipo $x_{n+1}=f(x_n)$ indipendentemente da $x_0$. Si vede subito che la successione $\left (d(z,x_n) \right )_{n\in\NN}$ è strettamente decrescente e non negativa, ammette dunque un limite $\rho\geq 0$. Per compattezza estraiamo una sottosuccessione $(x_{n(k)})_{k\in\NN}$ covergente ad un certo $\overline{x}$; per continuità è chiaro che $d(z,\overline{x})=\rho$. Si vede anche che la successione $(x_{n(k)+1})_{k\in\NN}$ converge a $f(\overline{x})$ e sempre per continuità $d(z,f(\overline{x}))=\rho$, ma se $z\neq\overline{x}$ potrei applicare l' ipotesi ed ottenere:
$$
\rho=d(z,\overline{x})<d(f(z),f(\overline{x}))=d(z,f(\overline{x}))=\rho\quad \Rightarrow \quad \mbox{assurdo}
$$
e dunque la tesi: $z=\overline{x}$. Inoltre non abbiamo usato mai il valore di $x_0$, da cui questa dimostrazione non dipende.
\ex{10.10} Per il problema $10.1$ esiste $a$ in $[0,1]$ punto fisso per $f$. Considero la successione $(a_n)_{n\in\NN}$ definita per ricorrenza da:
$$
a_0=a;\qquad a_{n+1}=g(a_n)
$$
osservo che tutti i punti di questa successione sono fissi per $f$, infatti induttivamente si ha:
$$
f(a_{k+1})=f(g(a_k))=g(f(a_k))=g(a_k)=a_{k+1}
$$
Supponiamo ora che per assurdo valga $f>g$ identicamente in $[0,1]$, da questo si ricava che $(a_n)_{n\in\NN}$ è strettamente descrescente infatti:
$$
a_k=f(a_k)>g(a_k)=a_{k+1}\geq 0
$$
da cui esiste $\overline{a}$ limite della successione $(a_n)_{n\in\NN}$. Per continuità di $f$ anche $\overline{a}$ è punto fisso per $f$, mentre per continuità di $g$ si ha che $\overline{a}$ è punto fisso anche per $g$, come si vede passando al limite la ricorrenza; ma allora $f(\overline{a})=g(\overline{a})$, assurdo perchè $f>g$.
